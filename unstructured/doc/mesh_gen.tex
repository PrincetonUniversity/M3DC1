\chapter{Mesh Preparation}
\section{Introduction}

The M3D-C1 requires a geometric model and a mesh that are the representation of the analysis domain. 
\begin{itemize}
\item PUMI is a parallel mesh infrastructure toolkit developed at SCOREC, RPI. For more information, visit http://www.scorec.rpi.edu/pumi
\item	Simmetrix provides a set of tools and libraries for engineering simulation including a state-of-art mesh generation. For more information, visit http://simmetrix.com.
\end{itemize}

The model file extensions referred in this document is the following
\begin{itemize}
\item .smd: Simmetrix-readable binary format model file  
\newline  The model generated with Simmetrix is saved in this format.
\item .dmg: PUMI-readable binary format model file
\newline	The model generated from PUMI mesh
\item	.txt: M3D-C1-readable ascii format model file 
\newline	The model is generated from mesh generation tool (See Section 2.2)
\item	arbitrary filename: M3D-C1-readable ascii format model file 
\newline  The first line of the file should contain five doubles (See Section 2.1)
\end{itemize}

The mesh file extensions referred in this document is the following.
\begin{itemize}
\item	.sms: Simmetrix-readable binary format mesh file
\begin{itemize}
\item	The mesh generated with Simmetrix is saved in this format.
\item	If a mesh is serial (1-part), the mesh file doesn’t have a number before the extension
\item	If a mesh is distributed (P-part, P$>$1), the mesh file has a number before the extension to represent the global part ID.
\end{itemize}
\item	.sms: ASCII format mesh file used in old PUMI stack
\begin{itemize}
\item	This format is not supported from January 2015
\item	If a mesh is serial (1-part), the mesh file doesn’t have a number before the extension
\item	If a mesh is distributed (P-part, P$>$1), the mesh file has a number before the extension to represent the global part ID.
\end{itemize}
\item	.smb: PUMI-readable binary format mesh file
\begin{itemize}
\item	This format is used in the current M3D-C1
\item	No matter if a mesh is serial (1-part) or distributed (P-part, P$>$1), the mesh file has a number before the extension to represent the global part ID.
\end{itemize}
\item .vtu/pvtu: binary format mesh file for visualization with paraview. For more information, visit http://paraview.org.
\end{itemize}

Model/Mesh requirements for M3D-C1 is the following
\begin{itemize}
\item	The model and mesh shall be generated as described in Section 2.1 and Section 2.2.
\item	The mesh file must be PUMI-readable .smb file. Note that a mesh file contains a “number” before the extension (.smb) to denote a global part ID.
\item	The model and mesh file must be present in the work directory
\item	The name of model and mesh file must be specified in “C1input” file in the work directory
\begin{itemize}
\item	mesh\_model = model\_file
\item	mesh\_filename = mesh\_file.smb (NOTE: do not specify a number before the file extension)
\end{itemize}
\item In a 2D run with P processes, there should be P mesh files with part ID from 0 to P-1
\item	In a 3D run with P*N processes where 2D mesh is distributed to P parts, 
\begin{itemize}
\item	there should be P mesh files with part ID from 0 to P-1
\item	in “C1input” file, specify “nplanes” to N (e.g. nplanes=8), where “nplanes” describes how many 2D mesh copies to be loaded
\item	the M3D-C1 code should be compiled with “3D=1, MAX\_PTS=60”.
\end{itemize}
\item The previous releases of M3D-C1 supported the ASCII format .sms mesh files, which are not supported any more. Therefore, any existing ASCII format .sms mesh files shall be converted to binary format (.smb). See Section 2.3 for how to convert the mesh format.
\end{itemize}

The rest of this section is organized as follows: Section 2.1 describes a simple mesh generation tool without Simmetrix libraries. Section 2.2 describes a mesh generation tool with Simmetrix libraries. Section 2.3 describes how to convert old PUMI mesh file (.sms) to the current mesh format (.smb). Section 2.4 presents how to split a mesh into a bigger number of parts. 


\section{Mesh Generation}
\section{Mesh Partitioning}

The program “split\_smb” allows the users to partition a P-part mesh into N parts (N$>$P). 

Location of “split\_smb” is the following:
\begin{itemize}
\item	on portalc7.pppl.gov: /p/tsc/m3dc1/lib/SCORECLib/rhel7/intel\_ver-openmpi\_ver/petsc\_ver/bin
\item	on cori.nersc.gov:  /global/project/projectdirs/mp288/cori/scorec/knl|hsw-petsc\_ver /bin
\item	on hydra.gate.rzg.mpg.de: /u/m3dc1/scorec/utilities/split\_smb
\end{itemize}

In order to split P-part mesh to N parts (N$>$P), run
\texttt{
mpirun –np N ./split\_smb input-mesh.smb output-mesh.smb X 
}
\begin{itemize}
\item	input-mesh should be .smb 
\item	output-mesh should be .smb
\item	N is the number of parts in the output mesh
\item	For a P-part input mesh, X must be N/P
\item	For both input and output mesh, do not specify a number before the file extension
\item	“split\_smb” will insert a number in the output mesh file. The number represents a global part ID.
\item	Make sure that the output mesh doesn’t have any empty part. Otherwise, the program crashes with the following error message:
\newline
APF warning: 1 empty parts
\newline
split\_smb: /u/sseol/develop/core/mds/mds.c:614: check\_ent: Assertion `e $>$= 0' failed.
\end{itemize}

Examples on portal:
\begin{enumerate}
\item To split a serial (1-part) mesh to 6 parts, run\\
 \texttt{mpirun –np 6 ./split\_smb struct-curveDomain.smb part.smb 6}
\begin{itemize}
\item	Input mesh: struct-curveDomain0.smb 
\item	Output mesh: part0.smb, part1.smb, part2.smb, part3.smb, part4.smb, part5.smb
\end{itemize}

\item To split a 2-part mesh to 6 parts, run
 \texttt{mpirun –np 6 ./split\_smb  struct-curveDomain.smb part.smb 3}
\begin{itemize}
\item	Input mesh: struct-curveDomain0.smb, struct-curveDomain1.smb
\item	Output mesh: part0.smb, part1.smb, part2.smb, part3.smb, part4.smb, part5.smb
\end{itemize}
\end{enumerate}

See ”readme.split\_smb” for detailed instructions and trouble shooting tips.


\section{Mesh Merging}

The program “collapse” allows the users to partition an N-part mesh into P parts (N$>$P)
\newline\newline
Location of ”collapse”
\begin{itemize}
\item	on portal.pppl.gov: \newline
/p/tsc/m3dc1/lib/SCORECLib/rhel6/intel\_ver-openmpi\_ver/petsc\_ver/bin
\item	on portalc7.pppl.gov: \newline
/p/tsc/m3dc1/lib/SCORECLib/rhel7/intel\_ver-openmpi\_ver/petsc\_ver/bin
\item	on cori.nersc.gov: \newline
 /global/project/projectdirs/mp288/cori/scorec/knl|hsw-petsc\_ver/bin
\end{itemize}

In order to merge N-part .smb mesh to P parts (P$>$0), run
\texttt{
	mpirun –np N ./collapse input-mesh.smb output-mesh.smb X 
}
\begin{itemize}
\item	input-mesh should be .smb 
\item	output-mesh should be .smb
\item	N is the number of parts in the input mesh
\item	For a P-part output mesh, X must be N/P
\item	For both input and output mesh, do not specify a number before the file extension
\item	“collapse” will insert a number in the output mesh file. The number represents a global part ID.
\end{itemize}

Example on portal:
\newline
In order to merge 4-part mesh into a serial (1-part) mesh, run
\texttt{mpirun –np 4 ./collapse part.smb serial.smb 4”
}
\begin{itemize}
\item	Input mesh: part0.smb, part1.smb, part2.smb, part3.smb
\item	Output mesh: serial0.smb
\end{itemize}

See ”readme.collapse” for detailed instructions and trouble shooting tips.


\section{Mesh Visualization}

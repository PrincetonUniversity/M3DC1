\chapter{SCOREC Software Interface}
This appendix chapter should document all SCOREC/ITAPS functions called from M3D-C$^1$. 
  In Fortran, the standard convention
is that a function does not modify any passed in variables and returns a value. 
In this appendix a function follows C/C++ conventions in both syntax and functionality. 
Functions listed below should be thought of as fortran subroutines as they are allowed
to modify the passed in parameters and do not return any value. 
The datatypes listed below in the C/C++ functions are:
\begin{description}
\item[int *] A 4 byte integer in Fortran.
\item[int[X]] A 4 byte integer array that should be allocated of at least length \textit{X}.
\item[char *] A character string in Fortran.
\item[double *] An 8 byte double precision/real*8 variable in Fortran.
\item[double[X]]  An 8 byte double precision/real*8 variable array in Fortran of at least length \textit{X}..
\item[crayptr] A Cray pointer in Fortran (e.g. pointer(a,b) where a is the Cray pointer and b is the Cray 
pointee).  The size of the Cray pointer is a 4 byte integer on 32 bit code and 8 byte integer (integer*8)
on 64 bit code.  
\end{description}
To make clear what is discussed below, a mesh vertex, edge, and face are the topological
entities of a mesh.  Many people associate nodes of a mesh with mesh vertices and elements with
the mesh faces for 2D meshes.  

The \textit{clearscorecdata} function cleans up everything associated with the SCOREC libraries.
Beyond this, there are 5 main sets of functions: model, mesh, ordering, vector and matrix.  The model
interface can be thought of as an independent object and is used to represent the geometric domain independent
of any mesh discretizing it.  The mesh only depends on the model that it discretizes.  The ordering
depends on the mesh and the model only and gives numberings of the degrees-of-freedom (DOFs) associated
with mesh entities (for the reduced quintic DOFs are only associated with mesh vertices).  The vector
and matrix interface depends mainly on the ordering but indirectly on the model and mesh as well.
 The two main functions
are to load a mesh and model which is used with the \textit{loadmesh} function below.  
\begin{center}
\begin{figure}
\centerline{\psfig{figure=./partitioned_mesh.eps,height=2.5in,angle=0}} %for epsfig
\caption{A partitioned mesh with the dofs numbered for the blue partition with one dof per node and iper=1 and jper=0.}\label{meshpartition} \end{figure}
\end{center}


\begin{itemize}
\item void loadmesh( char * modelfilename, char * meshfilename); This function
loads a model file called \textit{modelfilename} which should be ``struct.dmg'' 
and a mesh file called \textit{meshfilename} which should be ``struct-dmg.sms''.
\item void clearscorecdata(); This function clears all SCOREC software data.
\end{itemize}


\section{Model Interface}
\begin{itemize}
\item  void getmodeltags(int * bottom, int * right, int * top, int * left); This
function assumes that the domain is rectangular shaped and returns the four model entity 
IDs for the sides of the rectangle.  The main use of this function is for setting boundary 
conditions on DOFs associated with mesh entities classified on the boundary.  The mesh entity
classification is obtained through the \textit{zonedg} function 
for mesh edges and \textit{zonenod} function for
mesh nodes/vertices.
\item  void setperiodicinfo(int* xperiodic, int* zperiodic); Sets the flag that periodic
boundary conditions exist in the horizontal direction if \textit{xperiodic} is anything but zero
and that periodic
boundary conditions exist in the vertical direction if \textit{zperiodic} is anything but zero.
\item  void getperiodicinfo(int * xperiodic, int* zperiodic);	Gets the flag that periodic
boundary conditions exist.  If \textit{xperiodic} is one than they have been set in
the horizontal direction and zero if they have not been. If
 \textit{zperiodic} is one than they have been set in
the vertical direction and zero if they have not been.
\item  void getmincoord(double* xmin, double* zmin); Gets the minimum coordinate dimensions of the geometric domain. 
\textit{xmin} is the minimum horizontal coordinate value and \textit{zmin} is the minimum vertical coordinate value
returned by the function.
\item  void getmaxcoord(double* xmax, double* zmax); Gets the maximum coordinate dimensions of the geometric domain.  
\textit{xmax} is the maximum horizontal coordinate value and \textit{zmax} is the maximum vertical coordinate value
returned by the function.
\end{itemize}

\section{Mesh Interface}
\begin{itemize}
\item void numnod( int *NumNodes ) ; Returns the number of nodes, \textit{NumNodes}, in the processor's partition of the mesh.  Since
 some nodes  exist on partition boundaries (i.e. shared by multiple processors) summing \textit{NumNodes} over all
processors will result in more nodes than exist in the total mesh. The IDs of the nodes on a processor are numbered between
one and \textit{NumNodes}.
\item void xyznod( int *iNode , double X[3] ) ; For input node ID \textit{iNode}, this function returns the coordinate location in 
the \textit{X} array.  The \textit{X} array should be of at least length 3.
\item void numfac( int *NumFaces ) ; Returns the number of faces, \textit{NumFaces}, in the processor's partition of the mesh. 
For a 2D mesh, the faces will all be elements. Since
 any face for a 2D mesh  exists only on  one partition,  summing \textit{NumFaces} over all
processors will result in the total number of faces that exist in the mesh. The IDs of the faces on a processor are numbered between
one and \textit{NumFaces}.
\item void nodfac( int *iFace, int Nodes[4] ) ; For an input face ID, \textit{iFace}, this function returns the adjacent nodal IDs
in the \textit{Nodes} array.  
For mixed topology meshes the faces could be triangles or quadrilaterals so it is safest to have the \textit{Nodes} array to be of
length four.  The nodes IDs are returned in counter-clockwise order and it the face is a triangle then only the first three
values in the \textit{Nodes} array contain usable data.
\item void numglobalents(int * numnodes, int * numedges, int * numfaces, int * numregions); This function returns
the global number of nodes, edges, faces, and regions in the mesh in \textit{numnodes, numedges, numfaces}, and \textit{numregions}.
\item  void zonenod( int *iNod , int *iZone , int * iZoneDim); This function returns the model entity ID in 
\textit{iZone} and the model entity dimension in \textit{iZoneDim} for passed in mesh vertex \textit{iNod}.
\item void zonedg( int *iEdge , int *iZone , int * iZoneDim ); This function returns the model entity ID in 
\textit{iZone} and the model entity dimension in \textit{iZoneDim} for passed in mesh edge \textit{iEdge}.

\item void createsearchstructure(); This function sets up a search structure on each mesh partition.  
\item void deletesearchstructure(); This function deletes the search structure on each mesh partition.
\item void usesearchstructure(double* x, double* y, int* iFace); This function returns the face ID \textit{iFace}
of a mesh face that contains the given point \textit{(x,y)}.  The point \textit{(x,y)} is given in the
global coordinate system of the mesh.  The function returns -1 for \textit{iFace} if the point is not contained
in the domain of any mesh faces on the local processes partition.  Note  that multiple faces may contain
a given point (e.g. a mesh vertex coordinate).
\item void getelmsizes(int * iFace, double sizes[4]); For mesh face \textit{iFace}, this function returns an array
of doubles for the mesh size at each of the mesh vertices of the face.  
\end{itemize}





\section{Ordering Interface}
It should be noted that when using complex numbers that the orderings will act the same regardless
of the data type. As an example, for DOF number \textit{i}, the double precision array value will 
be stored at \textit{d(i)} and the complex array value will be stored at \textit{c(i)}.  Since
C and C++ have no intrinsic complex datatype, the complex arrays will be treated as double arrays
and the real part will be accessed at \textit{d[(i-1)*2]} and the imaginary part will be
accessed at \textit{d[(i-1)*2+1]}.

\begin{itemize}
\item   void createdofnumbering(int * numberingid, int * iper, int * jper,
                           int * dofspervertex, int * dofsperedge, 
                           int* dofsperface, int * dofsperregion, int * numdofs);
 This function creates a numbering with  ID \textit{numberingid}.  Currently, \textit{iper}
and \textit{jper} are not used as the periodic information is read in from the C1input file during the 
call to \textit{loadmesh}.
Eventually though if some orderings are periodic and others are not (e.g. a potential function)
than these options can be used to specify the difference in orderings. The other parameters
passed in (\textit{dofspervertex, dofsperedge, dofsperface,dofsperregion}) are to specify
how many dofs per mesh entity type (e.g. \textit{dofspervertex} would be 18 for a numvar=3 
vector and 0 for the other types.  The only returned value is \textit{numdofs} 
which is the total number of dofs associated with mesh entities that exist on the process that
the function is called on. This is also the allocated size of a vector on this process.  Note
though that a global sum of \textit{numdofs} on all processes will be greater than the total number
of global dofs unless it is a single process run. 

\item void deletedofnumbering(int * numberingid);
This function deletes the numbering associated with \textit{numberingid}.  It is an error
to query the SCOREC software with \textit{numberingid} unless another numbering is created
with this id.


\item  void entdofs(int * numberingid, int * meshentid, int * meshentdim,
               int * begindofnumber, int * enddofnumberplusone);
     For \textit{numberingid}, of topological dimension \textit{meshentdim} with mesh entity
ID \textit{meshentid} this function outputs \textit{begindofnumber} which is the beginning of
the dof array location (stored contiguously) and \textit{enddofnumberplusone} is one
beyond the last dof array location.  Note that the returned array index values are in Fortran
ordering convection such that the first entry of the array is at index 1.
For a numvar=1 ordering an example would be \textit{begindofnumber}=1
and \textit{enddofnumberplusone}=7. 

\item  void numdofs(int * numberingid, int * numdofs); This function returns the total number of 
DOFs associated with all mesh entities existing on a specific mesh partition.  For the example
in Figure \ref{meshpartition}, the blue partition would have \textit{numdofs} equal to 10 as both DOF 1
and 2 are periodic DOFs and only get counted once even though the number of nodes for this 
is 12.
  
\item  void numglobaldofs(int * numberingid, int * numglobaldofs);
This function returns the global number of DOFs \textit{numglobaldofs} for numbering \textit{numberingid}. 
 For the mesh in Figure \ref{meshpartition} \textit{numglobaldofs} would be 19.

\end{itemize}
\section{Vector Interface}
Cray pointers are used to simplify the allocation/reallocation of memory for the arrays used
for the vectors.  This appears to be the only way for C, C++, and Fortran to 
allocate and free memory regardless of where it is used or freed.  The syntax for creating
a Cray pointer in Fortan is:
\begin{verbatim}
  integer*8 ptr_x
  real*8 x(*)
  pointer(ptr_x, x)
\end{verbatim}
Here, \textit{ptr\_x} is the Cray pointer, which is an integer*8 in 64 bit code and an integer in 32 bit code,
and \textit{x} is the pointee.  Use \textit{ptr\_x = malloc(size)}
to allocate \textit{size} bytes and \textit{call free(ptr\_x)} to free the allocated memory.
Also, to ``nullify'' the Cray pointer do \textit{ptr\_x=0} but only if no memory is allocated to
the pointer to avoid memory leaks.  If the pointer has memory allocated for it, the memory is accessed through
\textit{x}.  For all SCOREC functions for both the vector and matrix interfaces the Cray pointer
should be passed in and not the pointee. On the C/C++ side the Cray pointer appears as
a double**.


\begin{itemize}
\item   void createppplvec(crayptr id, int * orderingid, int * type); This function creates a vector for Cray pointer
\textit{id} for numbering \textit{orderingid}. If the vector is for a real-valued array \textit{type}
should be set to 0 and if the vector is for  a complex-valued array \textit{type} should be set to 1. 
 If the Cray pointer has not been allocated it
must be set to 0 before calling this function.
\item   void deleteppplvec(crayptr id); This function deletes the vector, frees the memory associated 
with the Cray pointer and sets the Cray pointer to 0.
\item   void sumsharedvals(crayptr id);  This function assumes that the values inserted into the
array of the vector has only local contributions on each process and then this sums the parts
of the vector that are distributed across multiple processes and updates each of those values
to a single value.  An example of this would be to figure out how many mesh faces use each DOF.  To do this, 
first zero the vector, then 
iterate over all of the faces on each process, then iterate over each mesh vertex, get the DOFs
associated with each vertex and add 1 to each array indexed with the DOF number, then call this function.
For this operation, for DOF number 10 in Figure \ref{meshpartition} the number would be 7.
\item   void updatesharedvals(crayptr id);  This function assumes that the single process that ``owns'' a DOF
has the correct value and that other processes that need the DOF may or may not have the correct value.  With
this assumption, this function can be called to update the correct values for a DOF on processes
that need the dof but do not own it.  This is used primarily for distributing the DOF values after a matrix solve
from the process it was solved for on to the other processes that use the DOF.  This function might not be useful
for M3D-C$^1$ but is documented anyways.
\item   void checkveccreated(crayptr id, int * iscreated); This function checks whether a vector \textit{id} 
has already been created or not.  It returns 0 if it has not and 1 if it has been created but does no
action otherwise.  
\end{itemize}

\section{Matrix Interface}
\begin{itemize}
\item  void zerosuperlumatrix(int * matrixid, int * numberingid);  This function creates
an ``empty'' matrix with id \textit{matrixid} that uses the numbering \textit{numberingid}
for use with the SuperLU\_DIST solver.  If the matrix is real-valued then \textit{type}
should be set to 0 and if the matrix is complex-valued then \textit{type} should be set to 1. 
If the matrix has already been created then it just cleans out all components of the matrix.
\item  void zeropetscmatrix(int * matrixid, int * numberingid); This function creates
an ``empty'' matrix with id \textit{matrixid} that uses the numbering \textit{numberingid}
for use with the SuperLU\_DIST solver.  If the matrix is real-valued then \textit{type}
should be set to 0 and if the matrix is complex-valued then \textit{type} should be set to 1. 
If the matrix has already been created then it just cleans out all components of the matrix.
 \item void zeromultiplymatrix(int * matrixid, int * numberingid); This function creates
an ``empty'' matrix with id \textit{matrixid} that uses the numbering \textit{numberingid}
for use for multiplying with vectors.  If the matrix is real-valued then \textit{type}
should be set to 0 and if the matrix is complex-valued then \textit{type} should be set to 1. 
If the matrix has already been created then it just cleans out all components of the matrix.
\item  void insertval(int * matrixid, double/complex * val, int * valtype, int * row, 
		   int * column, int * operation);  This function inserts \textit{val} into
matrix \textit{matrixid} at \textit{(row,column)} in the matrix where \textit{row}
and \textit{column} come from the ordering.  The type of value to
 be inserted can be real (\textit{valtype}=0)
or complex (\textit{valtype}=1).  A real type can be inserted into a complex matrix but
a complex type cannot be inserted into a real matrix.  If \textit{operation}
is zero then the value overwrites any existing value, otherwise the value is to be added
to existing values for that matrix component.
\item  void setdiribc(int * matrixid, int * row);  For matrix \textit{matrixid}, this function
zeroes out all off-diagonal values for \textit{row} and set the diagonal value to unity.  The operation is
actually carried out during \textit{finalizematrix} so this function can be called before other values
are inserted into that row while still applying the correct operation.  For complex-valued arrays, 
only the real part of the diagonal is set to unity while the imaginary part is set to 0.  This function 
should be called on all processes that use the DOF number associated with the matrix row.
\item  void setgeneralbc(int * matrixid, int * row, int * numvals,
		     int  columninfo[numvals], double/complex vals[numvals], int * valtype);
This function sets multiple values for \textit{row} of \textit{matrixid}.  The number of values set is
\textit{numvals} and \textit{columninfo} specifies which columns to set the values for and 
\textit{vals} is the values to be set which must be in the same order as the \textit{columninfo} array.
 The type of values to
 be inserted can be real (\textit{valtype}=0)
or complex (\textit{valtype}=1).
This function 
should be called on all processes that use the DOF number associated with the matrix row.
\item  void finalizematrix(int * matrixid);  This function finalizes \textit{matrixid} such that no
more values can be inserted into the matrix and no more boundary conditions can be applied to the matrix.
\item  void solve(int * matrixid, crayptr rhs\_sol, int * ier);  For a matrix created with \textit{zeropetscmatrix}
or \textit{zerosuperlumatrix}, \textit{matrixid} is solved with input right-hand-side \textit{rhs\_sol}. 
The output  is also in \textit{rhs\_sol}.  If \textit{ier} is non-zero there were problems encountered during
the solve process.  The only way to properly mix real and complex value types is if the matrix is real-valued
and the vector is complex-valued.  Then the system is solved twice with a real and an imaginary right-hand-side.  
\item  void matrixvectormult(int * matrixid, crayptr inputvecid, crayptr outputvecid);  For a matrix created with \textit{zeromultiplymatrix}, this function can be called to multiply \textit{matrixid} with \textit{inputvecid}.  The output is returned in \textit{outputvecid}.  If either \textit{matrixid} or \textit{inputvecid}
is complex-valued, \textit{outputvecid} must also be complex-valued.
\item  void deletematrix(int * matrixid);  This function deletes \textit{matrixid}.
\item  void matrixrank(int* matrixid, int * rank);  This function returns the global rank of \textit{matrixid}
in \textit{rank}.  This value is the same as the returned value for \textit{numglobaldofs} for the matrix's
ordering.
\item  void initsolvers();  This function sets up the data structures required by the solvers.
\item  void finalizesolvers();  This function finalizes/destroys the data structures required by the solvers.
\item  void writematrixtofile(int * matrixid, int * fileid);  This function writes out the non-zero components
of \textit{matrixid} to a file that uses \textit{fileid} to indicate which file.
\end{itemize}
For the PETSc solver, the following two functions are defined in \textit{PETScInterface.c} and are designed
to give M3D-C$^1$ the ability to modify the PETSc solver parameters for a given \textit{matrixid}.  
The FortranMatrixID enumeration must match the parameters specified in the sparse module in \textit{M3Dmodules.f90}.
\begin{itemize}
\item  int setPETScMat(int matrixid, Mat * A);  This function can be used to set options for storing \textit{matrix}
in PETSc data structures.
\item int setPETScKSP(int matrixid, KSP * ksp, Mat * A); This function can be used to set the solver
parameters for \textit{matrixid} in PETSc.
\end{itemize}


% appendix on modules that are used on different machines that M3D-C1 is installed on
\chapter{Module use on different machines}
This appendix is meant to make sure that people do not have trouble determining which modules to use on 
which machines.  Note that this list may include modules that are not necessary for M3D-C$^1$.

\section{viz/m3d.pppl.gov}
The modules that I (Andy Bauer) have loaded on viz and m3d are  
 intel\_fc/9.0.033,
 intel\_cc/9.0.032,
 ncarg/4.4.1,
 hdf5/1.6.5,
 intel\_mkl/8.1.014,
   superlu\_dist\_2.0/20060102,
   intel\_vt/8.0.245,
   subversion/1.3.0,
   /parmetis/3.1,
  /autopack/1.3.2,
  petsc/2.3.3-p3,
  superlu\_3.0/20060102, and
  /zoltan/3.0 .

The commands that I use to load them are:
\begin{itemize}
\item      module load intel\_fc/9.0.033
\item       module load intel\_cc/9.0.032
\item       module load ncarg
\item       module load hdf5
\item       module load intel\_mkl/8.1.014
\item       module load superlu\_dist\_2.0/20060102
\item       module load intel\_vt
\item       module load subversion
\item       module load intel\_cc parmetis
\item       module load intel\_cc autopack
\item       module load petsc
\item       module load superlu\_3.0
\item       module load zoltan
\end{itemize}

\section{bassi.nersc.gov}
The modules that I (Andy Bauer) have loaded on bassi are 
null
   parmetis/3.1
   superlu\_dist/2.0\_64
   hdf5\_par/1.6.4
   netcdf/3.6.2
   ncar/4.4.2
   lapack/3.0
   petsc/2.3.3\_O
   zoltan/3.0 .

The commands that I use to load them are:
\begin{itemize}
\item module load parmetis
\item module load superlu\_dist
\item module load hdf5\_par
\item module load netcdf
\item module load ncar
\item module load petsc/2.3.3\_O
\item module load lapack
\item module load zoltan
\end{itemize}


\section{Generation and Management of M3D-C$^{1}$ Meshes}
\subsection{Introduction}

This section describes the methods used to generate M3D-C$^{1}$ meshes for Tokamak geometries. This process involves the use of specific software and interaction with several files that contain the associated geometry and mesh information.

The process of generating M3D-C$^{1}$ meshes involves the use of two software libraries. They are: 
\begin{itemize}
\item m3dc1\_meshgen coordinates the process of generating the an initial M3D-C$^{1}$
\item	Simmetrix provides a set of tools and libraries for engineering simulation including a state-of-art mesh generation. For more information, visit \href{http://simmetrix.com}{http://simmetrix.com}. The Simmetrix library is used to generate the M3D-C$^{1}$ meshes. The Simmetrix meshing library is used by m3dc1\_meshgen for the actual mesh generation. The Simmetrix GUI can be used after the execution of m3dc1\_meshgen to apply additional mesh controls for the generation of meshes with different mesh gradations. 
\item PUMI is a parallel mesh infrastructure toolkit developed at SCOREC, RPI. For more information, visit \href{http://www.scorec.rpi.edu/pumi}{http://www.scorec.rpi.edu/pumi}. The PUMI library is used to manage the mesh information as it is processed and within the M3D-C$^{1}$ code.
\end{itemize}

There are a number of files involved with housing the geometry and mesh information. 

The model file extensions referred in this document is the following
\begin{itemize}
\item \texttt{.smd}: Simmetrix-readable binary format model file  
\newline  The model generated with Simmetrix is saved in this format.
\item \texttt{.dmg}: PUMI-readable binary format model file
\newline	The model generated from PUMI mesh
\item	\texttt{.txt}: M3D-C$^{1}$-readable ascii format model file 
\newline	The model is generated from mesh generation tool (See Section~\ref{ch:mesh-gen})
\end{itemize}

The mesh file extensions referred in this document is the following.
\begin{itemize}
\item	\texttt{.sms}: Simmetrix-readable binary format mesh file
\begin{itemize}
\item	The mesh generated with Simmetrix is saved in this format.
\item	If a mesh is serial (1-part), the mesh file doesn't have a number before the extension
\item	If a mesh is distributed (\texttt{P}-part, P$>$1), the mesh file has a number before the extension to represent the global part ID.
\end{itemize}
\item	\texttt{.smb}: PUMI-readable binary format mesh file
\begin{itemize}
\item	This format is used in M3D-C$^{1}$ to import/export a mesh
\item	No matter if a mesh is serial (1-part) or distributed (\texttt{P}-part, P$>$1), the mesh file has a number before the extension to represent the global part ID.
\end{itemize}
\item \texttt{.vtu/pvtu}: binary format mesh file for visualization with Paraview. For more information, visit \href{http://paraview.org}{http://paraview.org}.
\end{itemize}

An overview of the Model/Mesh requirements for the M3D-C$^{1}$ mesh generation process are as follows:
\begin{itemize}
\item	The model and mesh shall be generated as described in Section~\ref{ch:mesh-gen}.
\item	The mesh file must be PUMI-readable \texttt{.smb} file. Note that a mesh file name contains a number before the extension (.smb) to denote a global part ID.
\item	The model and mesh file must be present in the work directory
\item	The name of model and mesh file must be specified in \texttt{C1input} file in the work directory
\begin{itemize}
\item	mesh\_model = model\_file
\item	mesh\_filename = mesh\_file.smb (NOTE: do not specify a number before the file extension)
\end{itemize}
\item In a 2D run with \texttt{P} processes, there should be \texttt{P} mesh files with part ID from \texttt{0} to \texttt{P-1}
\item	In a 3D run with \texttt{P$\times$N} processes where 2D mesh is distributed to \texttt{P} parts, 
\begin{itemize}
\item	there should be \texttt{P} mesh files with part ID from \texttt{0} to \texttt{P-1}
\item	in \texttt{C1input} file, specify \texttt{nplanes} to \texttt{N} (e.g. nplanes=8), where \texttt{nplanes} describes how many 2D mesh copies to be loaded
\item	the M3D-C$^{1}$ code should be compiled with options \texttt{"3D=1, MAX\_PTS=60"}.
\end{itemize}
\end{itemize}

The rest of this section is organized as follows: Section \ref{ch:mesh-gen} describes a mesh generation program \texttt{m3dc1\_meshgen}. Section \ref{ch:mesh-ptn} presents a mesh partitioning program \texttt{"split\_smb"} and \texttt{"collapse"} which changes the number of parts of the mesh. For how to visualize a mesh with \texttt{Paraview}, see Appendix~\ref{ch:app-paraview}.

%%%%%%%%%%%%%%%%%%%%%%%%%%%%%%%%%%%%%%%%%
\subsection{Mesh Generation}
\label{ch:mesh-gen}
%%%%%%%%%%%%%%%%%%%%%%%%%%%%%%%%%%%%%%%%%

The program \texttt{m3dc1\_meshgen} is used to define the tokamak cross section geometry and an initial mesh. Its execution requires a Simmetrix Licence, it is currently available only on PPPL Portal in the following location:
\newline\newline
\texttt{/p/tsc/m3dc1/lib/SCORECLib/rhel7/intel2019u3-openmpi4.0.3/petsc-3.13.5/bin}

%%%%%%%%%%%%%%%%%%%%%%%%%%%%%%%%%%%%%%%%%
\subsubsection{How to Run m3dc1\_meshgen}
%%%%%%%%%%%%%%%%%%%%%%%%%%%%%%%%%%%%%%%%%
\begin{enumerate}
\item Set environment variables for Simmetrix
\texttt{
\\module load intel/version openmpi/version
\\module load  simmodsuite/14.0-190402dev simmodeler/7.0-190402dev
\\module load paraview (for mesh visualization with .vtk files)
}
\item	Create an ascii file of arbitrary name that contains input parameters for model and mesh
\begin{itemize}
\item modelType: 0, 1, 2, 3, or 4
\begin{itemize}
\item Type 0: a parameterized vacuum region defined by five doubles for analytic expression. 
For five doubles $X_0$   $X_1$   $X_2$   $Z_0$   $Z_1$, vacuum boundary is defined by 
\begin{equation}
X = X_0 + X_1 cos(\theta + X_2*sin(\theta))
\end{equation}
\begin{equation}
Z =  Z_0 + Z_1 sin(\theta)
\end{equation}
\item	Type 1: a vacuum region defined by piece-wise linear points
\item	Type 2: a vacuum region defined by piece-wise polynomials
\item	Type 3: spline-fitted 3-region model (plasma, wall and vacuum)
\item	Type 4: spline-fitted 3-region model (plasma, wall, and vacuum) with inner \& outer boundary points to set resistive wall
\end{itemize}

\item reorder: if 1, reorder PUMI mesh based on adjacency (default: 0) and generate vtk folders for mesh visualization. The mesh before and after reodering is saved in \texttt{original-mesh.vtk} and \texttt{reordered-mesh.vtk}, respectively. Note that the element order of Simmetrix mesh is not affected.
\item inFile: (modelType 0) not required
(modelType 1 and 2) geometry file describing the vacuum (modelType 3 and 4) geometry file describing the inner plasma wall
\item bdryFile: (modelType 0-3) not required
 (modelType 4) geometry file describing the outer plasma wall
\item outFile: output file name to save model and mesh
\item meshSize: relative mesh size for each region (default 0.05)
\newline
	for modelType 3, set three doubles for plasma, resistive, vacuum, respectively
\item useVacuumParams: for modelType 0 or 3, if 1, use parameterized vacuum wall (default 0)
\item vacuumParams: five doubles to describe parameterized vacuum wall. Required if useVacuumParams=1.
\item adjustVacuumParams: for modelType 0 or 3, if 1, multiply coordinates and parametric values of nodes on vacuum wall by vacuumFactor. Valid only if useVacuumParams=1 (default 0)
\item vacuumFactor: for modelType 0 or 3, an optional double value used to multiply coordinates and parametric values of nodes on vacuum wall when adjustVacuumParams=1. Valid only if adjustVacuumParams=1 (default 2$\times$PI)
\item numVacuumPts: optional \# interpolation points on parameterized vacuum wall. Valid only if useVacuumParams=1 (default 20)
\item meshGradationRate: for modelType 3 or 4, optional mesh gradation rate (default: 0.3). This value should be greater than or equal to 0.3. Otherwise the mesh will be fine everywhere.
\item resistive-width: for modelType 3, the width of resistive wall. If resistive-width=0, only plasma region is created (default 0.02)
\item plasma-offsetX: for modelType 3, the offset in x direction to the left (default 0.0)
\item plasma-offsetY: for modelType 3, the offset in y direction to the bottom (default 0.0)
\item vacuum-width: for modelType 3 or 4, the width of vacuum region (default 2.5)
\item vacuum-height: for modelType 3 or 4, the height of vacuum region (default 4.0) 
\end{itemize}

\item Run \texttt{m3dc1\_meshgen}
\begin{itemize}
\item input file, inFile, and bdryFile (if applicable) should be in the work folder
\item \texttt{argv[1]}: the ascii file created in Step 2
\item The output model is saved in three formats
  \begin{itemize}
  \item	$M3D-C^1$-readable \texttt{.txt}
  \item	Simmetrix-readable file \texttt{.smd} and 
  \item	PUMI-readable \texttt{.dmg}
    \begin{itemize}
    \item[$\triangleright$] For modelType 0-2, the model is saved in \texttt{outFile.*}
    \item[$\triangleright$] For modelType 3 with resistive width \texttt{R}, vacuum-width \texttt{W} and vacuum-height \texttt{H}, the model is saved in \texttt{outFile-R-W-H.*}.
    \item[$\triangleright$] For modelType 4 with vacuum-width \texttt{W} and vacuum-height \texttt{H}, the model is saved in \texttt{outFile-W-H.*}.
    \end{itemize}
  \end{itemize}
\item The output mesh is saved in three formats
  \begin{itemize}
  \item Simmetrix-readable\texttt{.sms}
  \item $M3D-C^1$/PUMI readable \texttt{.smb}
  \item Paraview 
    \begin{itemize}
    \item[$\triangleright$] For modelType 0-2 with \# mesh faces \texttt{F}, 
      \begin{itemize}
      \item[-] if \texttt{F} $>$ 1000, the mesh is saved in \texttt{outFile-(F/1000).*}
      \item[-] if \texttt{F} $<$ 1000, the mesh is saved in \texttt{outFile-F.*}
      \end{itemize}
    \item[$\triangleright$] For modelType 3 with \# mesh faces \texttt{F}, resistive width \texttt{R}, vacuum-width \texttt{W}, vacuum-height \texttt{H},
      \begin{itemize}
      \item[-] if \texttt{F} $>$ 1000, the mesh is saved in \texttt{outFile-R-W-H-(F/1000).*}
      \item[-] if \texttt{F} $<$ 1000, the mesh is saved in \texttt{outFile-R-W-H-F.*}
      \end{itemize}
    \item[$\triangleright$] For modelType 4 with \# mesh faces \texttt{F}, vacuum-width \texttt{W} and vacuum-height \texttt{H},
      \begin{itemize}
      \item[-] if \texttt{F} $>$ 1000, the mesh is saved in \texttt{outFile-W-H-(F/1000).*}
      \item[-] if \texttt{F} $<$ 1000, the mesh is saved in \texttt{outFile-W-H-F.*}
      \end{itemize}
    \end{itemize}
  \end{itemize}
\end{itemize}

\item If the initial mesh is not good enough, run \texttt{simmodeler} to generate a mesh with more meshing controls. See Section~\ref{ch:simmodeler} for detailed instructions.
\item In order to load the model and mesh, locate them in your work directory and modify C1input parameters
\end{enumerate}
See \texttt{readme.m3dc1\_meshgen} for detailed instructions and trouble shooting tips.


%%%%%%%%%%%%%%%%%%%%%%%%%%%%%%%%%%%%%%%%%
\subsubsection{Example: Type 0 (parameterized vacuum)}
%%%%%%%%%%%%%%%%%%%%%%%%%%%%%%%%%%%%%%%%%

See the example input file \texttt{"analytic-input"}.


%%%%%%%%%%%%%%%%%%%%%%%%%%%%%%%%%%%%%%%%%
\subsubsection{Example: Type 1 (piece-wise linear vacuum)}
%%%%%%%%%%%%%%%%%%%%%%%%%%%%%%%%%%%%%%%%%

%%%%%%%%%%%%%%%%%%%%%%%%%%%%%%%%%%%%%%%%%
\subsubsection{Example: Type 2 (piece-wise polynomial vacuum)} 
%%%%%%%%%%%%%%%%%%%%%%%%%%%%%%%%%%%%%%%%%

See the example input file \texttt{"poly-input"}.

%%%%%%%%%%%%%%%%%%%%%%%%%%%%%%%%%%%%%%%%%
\subsubsection{Example: Type 3 (three-regions with inner wall points)}
%%%%%%%%%%%%%%%%%%%%%%%%%%%%%%%%%%%%%%%%%
See the example input file \texttt{"circle-input"}.

%%%%%%%%%%%%%%%%%%%%%%%%%%%%%%%%%%%%%%%%%
\subsubsection{Example: Type 4: (three-regions with inner $\&$ outer wall points)}
%%%%%%%%%%%%%%%%%%%%%%%%%%%%%%%%%%%%%%%%%
See the example input file \texttt{"bdry-input"}.

%%%%%%%%%%%%%%%%%%%%%%%%%%%%%%%%%%%%%%%%%
\subsubsection{Example: Multiple-regions with inner $\&$ outer wall points}
%%%%%%%%%%%%%%%%%%%%%%%%%%%%%%%%%%%%%%%%%
See the example input file \texttt{"multi-input"}.

%%%%%%%%%%%%%%%%%%%%%%%%%%%%%%%%%%%%%%%%%
\subsubsection{Example: Equilibrium}
%%%%%%%%%%%%%%%%%%%%%%%%%%%%%%%%%%%%%%%%%

The program \texttt{read\_jsolver} generates equilibrium and stores in the file \texttt{POLAR}. Given the input file \texttt{POLAR}, \texttt{m3dc1\_meshgen} generates the following files:
\begin{itemize}
\item	model.dmg: PUMI-readable model file
\item	model.txt: M3DC1-readable model file
\item	mesh0.smb: PUMI/M3DC1-readable mesh file
\item	mesh.vtk: Paraview data files
\item	norm\_curv: ascii file containing nodes' normal/curvature information
\end{itemize} 

%%%%%%%%%%%%%%%%%%%%%%%%%%%%%%%%%%%%%%%%%
\subsection{Mesh Control with SimModeler}
SimModeler is a graphical user interface to the Simmetrix geometry and mesh generation software. In cases where the currently available capabilities of m3dc1\_meshgen do not provide a satisfactory mesh, SimModeler can be used to apply alternative mesh control information to the Tokamak cross section geometry to generate different meshes. The information below indicates the application of a subset of the mesh controls that can be applied. For additional information of the full range of SimModeler mesh control options see: ********** FILL IN POINTER TO SIMMETRIX DOCUMENTATION *****
\label{ch:simmodeler}
%%%%%%%%%%%%%%%%%%%%%%%%%%%%%%%%%%%%%%%%%
(Contributed by D. Pfefferle on 4/27/16) On PPPL Portal, load a module \texttt{simmodeler} and run it.
\begin{enumerate}
\item From the menu \texttt{"File$\rightarrow$Open Model"}, load a model file (\texttt{.smd}) generated by \texttt{m3dc1\_meshgen}
\item In the upper panel, in the views section, click on \texttt{Front} to view the model, then go to \texttt{Meshing} tab
\item Select outer region, click \texttt{+} in \texttt{Mesh Attributes} and select \texttt{Mesh Size$\rightarrow$relative}. 
Enter a value (typically 0.1)
\item Select wall region, click \texttt{+} in \texttt{Mesh Attributes} and select \texttt{Mesh Size$\rightarrow$relative}. 
Enter a value (typically 0.02)
\item Select inner region, click \texttt{+} in \texttt{Mesh Attributes} and select \texttt{Mesh Size$\rightarrow$relative}. 
Enter a value (typically 0.04). Here, one can already generate the mesh by clicking on \texttt{Generate Mesh} and verify if the mesh sizes are suitable 
\item 	Select both inner and wall regions (holding shift key), click \texttt{+} in \texttt{Mesh Attributes} and select \texttt{Mesh Size$\rightarrow$relative}. Enter a function, e.g. \texttt{0.01$\times$abs(\$y+1.5)\^{}2+0.004} to specify an anisotropic mesh density on top of previous settings
\\ There are many available parameters for fine-tuning the mesh density.  For example, \texttt{Mesh Curvature Refinement} with parameter packs more elements near the edges of the resistive wall. 
\item \texttt{Generate Mesh} and \texttt{Show Mesh} to view result in new windows
\item If the result is satisfactory, from the menu \texttt{File$\rightarrow$Save Mesh}, give it a meaningful name with the extension \texttt{.sms}. The original model file \texttt{.smd} has been automatically saved by the program with your mesh modifications.
\item Close \texttt{simmodeler} then it will release a license. Until you quit Simmodeler, no one cannot run neither \texttt{m3dc1\_meshgen} nor \texttt{simmodeler}.
\item Copy the \texttt{.txt, .smd} and \texttt{.sms} files to the simulation directory and run the following splitting routine to obtain PUMI-readable \texttt{.smb} mesh files.
\newline\newline
\texttt{/p/tsc/C1/m3dc1-sunfire.r6-1.5/bin/part\_mesh.sh model\_file.smd mesh\_file.sms X}, 
where \texttt{X} is the number of parts you need in the \texttt{.smb} mesh.
\item Modify the \texttt{C1input} file accordingly
\newline\newline
\texttt{mesh\_filename = `part.smb'
\\
mesh\_model = `filename.txt'
}
\end{enumerate}

%%%%%%%%%%%%%%%%%%%%%%%%%%%%%%%%%%%%%%%%%
\subsection{Mesh Partitioning}
\label{ch:mesh-ptn}
%%%%%%%%%%%%%%%%%%%%%%%%%%%%%%%%%%%%%%%%%

\subsubsection{Splitting}

The program \texttt{split\_smb} increases the number of parts in a mesh from \texttt{P} to \texttt{N} (\texttt{P$<$N}). 
In each machine, the program \texttt{split\_smb} is availble in \texttt{\$SCOREC\_UTIL\_DIR} provided in \texttt{hostname.mk} file.

In order to split \texttt{P}-part mesh to \texttt{N} parts (\texttt{N$>$P}), run
\texttt{"mpirun -np N ./split\_smb input-mesh(.smb) output-mesh(.smb) X"}
\begin{itemize}
\item	the file extension of input-mesh should be .smb 
\item	the file extension of output-mesh should be .smb
\item	\texttt{N} is the number of parts in the output mesh
\item	For a \texttt{P}-part input mesh, \texttt{X} must be \texttt{N/P}
\item	For both input and output mesh, do not specify a number before the file extension
\item	\texttt{split\_smb} will insert a number in the output mesh file. The number represents a global part ID.
\item	Make sure that the output mesh doesn't have any empty part. Otherwise, the program crashes with the following error message:
\newline
\texttt{APF warning: 1 empty parts}
\newline
\texttt{split\_smb: \ldots/mds/mds.c:614: check\_ent: Assertion `e $>$= 0' failed}
\end{itemize}

Examples on portal:
\begin{enumerate}
\item To split a serial (1-part) mesh to 6 parts, run\\
 \texttt{"mpirun -np 6 ./split\_smb struct-curveDomain.smb part.smb 6"}
\begin{itemize}
\item	Input mesh: struct-curveDomain0.smb 
\item	Output mesh: part0.smb, part1.smb, part2.smb, part3.smb, part4.smb, part5.smb
\end{itemize}

\item To split a 2-part mesh to 6 parts, run
 \texttt{"mpirun -np 6 ./split\_smb  struct-curveDomain.smb part.smb 3"}
\begin{itemize}
\item	Input mesh: struct-curveDomain0.smb, struct-curveDomain1.smb
\item	Output mesh: part0.smb, part1.smb, part2.smb, part3.smb, part4.smb, part5.smb
\end{itemize}
\end{enumerate}

See \texttt{readme.split\_smb} for detailed instructions and trouble shooting tips.

%%%%%%%%%%%%%%%%%%%%%%%%%%%%%%%%%%%%%%%%%
\subsubsection{Mesh Merging}
\label{ch:mesh-mg}
%%%%%%%%%%%%%%%%%%%%%%%%%%%%%%%%%%%%%%%%%

The program \texttt{collapse} decreases the number of parts in a mesh from \texttt{N} to \texttt{P} (\texttt{P$<$N}). 
In each machine, the program \texttt{collapse} is availble in \texttt{\$SCOREC\_UTIL\_DIR} provided in \texttt{hostname.mk} file.

In order to merge \texttt{N}-part .smb mesh to \texttt{P} parts (\texttt{P$>$0}), run
\texttt{"mpirun -np N ./collapse input-mesh(.smb) output-mesh(.smb) X"}
\begin{itemize}
\item	the file extension of input-mesh should be .smb 
\item	the file extension of output-mesh should be .smb
\item	\texttt{N} is the number of parts in the input mesh
\item	For a \texttt{P}-part output mesh, \texttt{X} must be \texttt{N/P}
\item	For both input and output mesh, do not specify a number before the file extension
\item	\texttt{collapse} will insert a number in the output mesh file. The number represents a global part ID.
\end{itemize}

Example on portal:
\newline
In order to merge 4-part mesh into a serial (1-part) mesh, run
\texttt{"mpirun -np 4 ./collapse part.smb serial.smb 4"}
\begin{itemize}
\item	Input mesh: part0.smb, part1.smb, part2.smb, part3.smb
\item	Output mesh: serial0.smb
\end{itemize}

See \texttt{readme.collapse} for detailed instructions and trouble shooting tips.

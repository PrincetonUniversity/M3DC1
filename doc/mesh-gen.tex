\section{Mesh Preparation}
\subsection{Introduction}

The M3D-C$^{1}$ requires a geometric model and a mesh that are the representation of the analysis domain. 
\begin{itemize}
\item PUMI is a parallel mesh infrastructure toolkit developed at SCOREC, RPI. For more information, visit \href{http://www.scorec.rpi.edu/pumi}{http://www.scorec.rpi.edu/pumi}.
\item	Simmetrix provides a set of tools and libraries for engineering simulation including a state-of-art mesh generation. For more information, visit \href{http://simmetrix.com}{http://simmetrix.com}.
\end{itemize}

The model file extensions referred in this document is the following
\begin{itemize}
\item \texttt{.smd}: Simmetrix-readable binary format model file  
\newline  The model generated with Simmetrix is saved in this format.
\item \texttt{.dmg}: PUMI-readable binary format model file
\newline	The model generated from PUMI mesh
\item	\texttt{.txt}: M3D-C$^{1}$-readable ascii format model file 
\newline	The model is generated from mesh generation tool (See Section~\ref{ch:mesh-gen})
\end{itemize}

The mesh file extensions referred in this document is the following.
\begin{itemize}
\item	\texttt{.sms}: Simmetrix-readable binary format mesh file
\begin{itemize}
\item	The mesh generated with Simmetrix is saved in this format.
\item	If a mesh is serial (1-part), the mesh file doesn't have a number before the extension
\item	If a mesh is distributed (\texttt{P}-part, P$>$1), the mesh file has a number before the extension to represent the global part ID.
\end{itemize}
\item	\texttt{.smb}: PUMI-readable binary format mesh file
\begin{itemize}
\item	This format is used in M3D-C$^{1}$ to import/export a mesh
\item	No matter if a mesh is serial (1-part) or distributed (\texttt{P}-part, P$>$1), the mesh file has a number before the extension to represent the global part ID.
\end{itemize}
\item \texttt{.vtu/pvtu}: binary format mesh file for visualization with Paraview. For more information, visit \href{http://paraview.org}{http://paraview.org}.
\end{itemize}

Model/Mesh requirements for M3D-C$^{1}$ is the following
\begin{itemize}
\item	The model and mesh shall be generated as described in Section~\ref{ch:mesh-gen}.
\item	The mesh file must be PUMI-readable \texttt{.smb} file. Note that a mesh file name contains a number before the extension (.smb) to denote a global part ID.
\item	The model and mesh file must be present in the work directory
\item	The name of model and mesh file must be specified in \texttt{C1input} file in the work directory
\begin{itemize}
\item	mesh\_model = model\_file
\item	mesh\_filename = mesh\_file.smb (NOTE: do not specify a number before the file extension)
\end{itemize}
\item In a 2D run with \texttt{P} processes, there should be \texttt{P} mesh files with part ID from \texttt{0} to \texttt{P-1}
\item	In a 3D run with \texttt{P$\times$N} processes where 2D mesh is distributed to \texttt{P} parts, 
\begin{itemize}
\item	there should be \texttt{P} mesh files with part ID from \texttt{0} to \texttt{P-1}
\item	in \texttt{C1input} file, specify \texttt{nplanes} to \texttt{N} (e.g. nplanes=8), where \texttt{nplanes} describes how many 2D mesh copies to be loaded
\item	the M3D-C$^{1}$ code should be compiled with options \texttt{"3D=1, MAX\_PTS=60"}.
\end{itemize}
\end{itemize}

The rest of this section is organized as follows: Section \ref{ch:mesh-gen} describes a mesh generation program \texttt{m3dc1\_meshgen}. Section \ref{ch:mesh-ptn} presents a mesh partitioning program \texttt{"split\_smb"} and \texttt{"collapse"} which changes the number of parts of the mesh. For how to visualize a mesh with \texttt{Paraview}, see Appendix~\ref{ch:app-paraview}.

%%%%%%%%%%%%%%%%%%%%%%%%%%%%%%%%%%%%%%%%%
\subsection{Mesh Generation}
\label{ch:mesh-gen}
%%%%%%%%%%%%%%%%%%%%%%%%%%%%%%%%%%%%%%%%%

The program \texttt{m3dc1\_meshgen} generates a mesh. As it requires a Simmetrix Licence, it is currently available only on PPPL Portal in the following location:
\newline\newline
\texttt{/p/tsc/m3dc1/lib/SCORECLib/rhel7/intel2019u3-openmpi4.0.3/petsc-3.13.5/bin}
\newline\newline

\subsubsection{How to Run m3dc1\_meshgen}
\begin{enumerate}
\item Set environment variables for Simmetrix
\newline
\\module load intel/version openmpi/version
\\module load  simmodsuite/14.0-190402dev simmodeler/7.0-190402dev
\\module load paraview (for mesh visualization with .vtk files)
\item	Create an ascii file of arbitrary name that contains input parameters for model and mesh
\begin{itemize}
\item modelType: 0, 1, 2, 3, or 4
\item reorder: if 1, reorder PUMI mesh based on adjacency (default: 0) and generate vtk folders for mesh visualization. The mesh before and after reodering is saved in \texttt{original-mesh.vtk} and \texttt{reordered-mesh.vtk}, respectively. Note that the element order of Simmetrix mesh is not affected.
\item inFile: (modelType 0) not required
(modelType 1 and 2) geometry file describing the vacuum (modelType 3 and 4) geometry file describing the inner plasma wall
\item bdryFile: (modelType 0-3) not required
 (modelType 4) geometry file describing the outer plasma wall
\item outFile: output file name to save model and mesh
\item meshSize: relative mesh size for each region (default 0.05)
\newline
	for modelType 3, set three doubles for plasma, resistive, vacuum, respectively
\item useVacuumParams: for modelType 0 or 3, if 1, use parameterized vacuum wall (default 0)
\item vacuumParams: five doubles to describe parameterized vacuum wall. Required if useVacuumParams=1.
\item adjustVacuumParams: for modelType 0 or 3, if 1, multiply coordinates and parametric values of nodes on vacuum wall by vacuumFactor. Valid only if useVacuumParams=1 (default 0)
\item vacuumFactor: for modelType 0 or 3, an optional double value used to multiply coordinates and parametric values of nodes on vacuum wall when adjustVacuumParams=1. Valid only if adjustVacuumParams=1 (default 2$\times$PI)
\item numVacuumPts: optional \# interpolation points on parameterized vacuum wall. Valid only if useVacuumParams=1 (default 20)
\item meshGradationRate: for modelType 3 or 4, optional mesh gradation rate (default: 0.3). This value should be greater than or equal to 0.3. Otherwise the mesh will be fine everywhere.
\item resistive-width: for modelType 3, the width of resistive wall. If resistive-width=0, only plasma region is created (default 0.02)
\item plasma-offsetX: for modelType 3, the offset in x direction to the left (default 0.0)
\item plasma-offsetY: for modelType 3, the offset in y direction to the bottom (default 0.0)
\item vacuum-width: for modelType 3 or 4, the width of vacuum region (default 2.5)
\item vacuum-height: for modelType 3 or 4, the height of vacuum region (default 4.0) 
\end{itemize}

Four input files are available for your reference: 
\begin{itemize}
\item	analytic-input (modelType 0)
\item	poly-input (modelType 2)
\item	circle-input (modelType 3)
\item	bdry-input (modelType 4)
\end{itemize}

\item Run \texttt{m3dc1\_meshgen} input file, inFile, and bdryFile (if applicable) should be in the work folder
\item If the initial mesh is not good enough, run \texttt{simmodeler} to generate a mesh with more meshing controls. 
\item In order to load the model and mesh, locate them in your work directory and modify C1input parameters 
\end{enumerate}

See \texttt{readme.m3dc1\_meshgen} for detailed instructions and trouble shooting tips.

%%%%%%%%%%%%%%%%%%%%%%%%%%%%%%%%%%%%%%%%%
\subsubsection{Example: Parameterized vacuum}
%%%%%%%%%%%%%%%%%%%%%%%%%%%%%%%%%%%%%%%%%

%%%%%%%%%%%%%%%%%%%%%%%%%%%%%%%%%%%%%%%%%
\subsubsection{Example: Piece-wise linear vacuum}
%%%%%%%%%%%%%%%%%%%%%%%%%%%%%%%%%%%%%%%%%

%%%%%%%%%%%%%%%%%%%%%%%%%%%%%%%%%%%%%%%%%
\subsubsection{Example: Piece-wise polynomial vacuum} 
%%%%%%%%%%%%%%%%%%%%%%%%%%%%%%%%%%%%%%%%%

%%%%%%%%%%%%%%%%%%%%%%%%%%%%%%%%%%%%%%%%%
\subsubsection{Example: Three-regions with inner wall points}
%%%%%%%%%%%%%%%%%%%%%%%%%%%%%%%%%%%%%%%%%

%%%%%%%%%%%%%%%%%%%%%%%%%%%%%%%%%%%%%%%%%
\subsubsection{Example: Three-regions with inner/outer wall points}
%%%%%%%%%%%%%%%%%%%%%%%%%%%%%%%%%%%%%%%%%

%%%%%%%%%%%%%%%%%%%%%%%%%%%%%%%%%%%%%%%%%
\subsubsection{Example: Multiple-regions with inner/outer wall points}
%%%%%%%%%%%%%%%%%%%%%%%%%%%%%%%%%%%%%%%%%

%%%%%%%%%%%%%%%%%%%%%%%%%%%%%%%%%%%%%%%%%
\subsubsection{Example: Equilibrium}
%%%%%%%%%%%%%%%%%%%%%%%%%%%%%%%%%%%%%%%%%

The program \texttt{read\_jsolver} generates equilibrium and stores in the file \texttt{POLAR}. Given the input file \texttt{POLAR}, \texttt{m3dc1\_meshgen} generates the following files:
\begin{itemize}
\item	model.dmg: PUMI-readable model file
\item	model.txt: M3DC1-readable model file
\item	mesh0.smb: PUMI/M3DC1-readable mesh file
\item	mesh.vtk: Paraview data files
\item	norm\_curv: ascii file containing nodes' normal/curvature information
\end{itemize}

%%%%%%%%%%%%%%%%%%%%%%%%%%%%%%%%%%%%%%%%%
\subsection{Mesh Partitioning}
\label{ch:mesh-ptn}
%%%%%%%%%%%%%%%%%%%%%%%%%%%%%%%%%%%%%%%%%

\subsubsection{Splitting}

The program \texttt{split\_smb} increases the number of parts in a mesh from \texttt{P} to \texttt{N} (\texttt{P$<$N}). 
In each machine, the program \texttt{split\_smb} is availble in \texttt{\$SCOREC\_UTIL\_DIR} provided in \texttt{hostname.mk} file.

In order to split \texttt{P}-part mesh to \texttt{N} parts (\texttt{N$>$P}), run
\texttt{"mpirun -np N ./split\_smb input-mesh(.smb) output-mesh(.smb) X"}
\begin{itemize}
\item	the file extension of input-mesh should be .smb 
\item	the file extension of output-mesh should be .smb
\item	\texttt{N} is the number of parts in the output mesh
\item	For a \texttt{P}-part input mesh, \texttt{X} must be \texttt{N/P}
\item	For both input and output mesh, do not specify a number before the file extension
\item	\texttt{split\_smb} will insert a number in the output mesh file. The number represents a global part ID.
\item	Make sure that the output mesh doesn't have any empty part. Otherwise, the program crashes with the following error message:
\newline
\texttt{APF warning: 1 empty parts}
\newline
\texttt{split\_smb: \ldots/mds/mds.c:614: check\_ent: Assertion `e $>$= 0' failed}
\end{itemize}

Examples on portal:
\begin{enumerate}
\item To split a serial (1-part) mesh to 6 parts, run\\
 \texttt{"mpirun -np 6 ./split\_smb struct-curveDomain.smb part.smb 6"}
\begin{itemize}
\item	Input mesh: struct-curveDomain0.smb 
\item	Output mesh: part0.smb, part1.smb, part2.smb, part3.smb, part4.smb, part5.smb
\end{itemize}

\item To split a 2-part mesh to 6 parts, run
 \texttt{"mpirun -np 6 ./split\_smb  struct-curveDomain.smb part.smb 3"}
\begin{itemize}
\item	Input mesh: struct-curveDomain0.smb, struct-curveDomain1.smb
\item	Output mesh: part0.smb, part1.smb, part2.smb, part3.smb, part4.smb, part5.smb
\end{itemize}
\end{enumerate}

See \texttt{readme.split\_smb} for detailed instructions and trouble shooting tips.

%%%%%%%%%%%%%%%%%%%%%%%%%%%%%%%%%%%%%%%%%
\subsubsection{Mesh Merging}
\label{ch:mesh-mg}
%%%%%%%%%%%%%%%%%%%%%%%%%%%%%%%%%%%%%%%%%

The program \texttt{collapse} decreases the number of parts in a mesh from \texttt{N} to \texttt{P} (\texttt{P$<$N}). 
In each machine, the program \texttt{collapse} is availble in \texttt{\$SCOREC\_UTIL\_DIR} provided in \texttt{hostname.mk} file.

In order to merge \texttt{N}-part .smb mesh to \texttt{P} parts (\texttt{P$>$0}), run
\texttt{"mpirun -np N ./collapse input-mesh(.smb) output-mesh(.smb) X"}
\begin{itemize}
\item	the file extension of input-mesh should be .smb 
\item	the file extension of output-mesh should be .smb
\item	\texttt{N} is the number of parts in the input mesh
\item	For a \texttt{P}-part output mesh, \texttt{X} must be \texttt{N/P}
\item	For both input and output mesh, do not specify a number before the file extension
\item	\texttt{collapse} will insert a number in the output mesh file. The number represents a global part ID.
\end{itemize}

Example on portal:
\newline
In order to merge 4-part mesh into a serial (1-part) mesh, run
\texttt{"mpirun -np 4 ./collapse part.smb serial.smb 4"}
\begin{itemize}
\item	Input mesh: part0.smb, part1.smb, part2.smb, part3.smb
\item	Output mesh: serial0.smb
\end{itemize}

See \texttt{readme.collapse} for detailed instructions and trouble shooting tips.

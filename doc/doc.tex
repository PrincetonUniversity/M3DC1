\newcommand{\dt}{\ensuremath{\delta t}}
\newcommand{\ddt}[1]{\frac{\partial #1}{\partial t}}
\newcommand{\thimp}{\ensuremath{\theta}}

\newcommand{\order}[1]{\ensuremath{\mathcal{O}(#1)}}

\renewcommand{\vec}[1]{\ensuremath{\mathbf{#1}}}
\newcommand{\tensor}[1]{\mathsf{#1}}
\newcommand{\tor}{\varphi}              % toroidal coordinate
\newcommand{\A}{\vec{A}}
\newcommand{\B}{\vec{B}}
\newcommand{\E}{\vec{E}}
\newcommand{\R}{\vec{R}}
\newcommand{\x}{\vec{x}}
\renewcommand{\r}{R}
\renewcommand{\v}{\vec{v}}
\renewcommand{\u}{\vec{u}}
\newcommand{\F}{\vec{F}}
\renewcommand{\j}{\vec{J}}
\newcommand{\q}{\vec{q}}
\newcommand{\g}{\vec{g}}
\newcommand{\jn}{\frac{\j}{n}}
\renewcommand{\P}{\tensor{\Pi}}
\renewcommand{\b}{\vec{b}}
\newcommand{\W}{\tensor{W}}
\newcommand{\codename}{\textsc{M3D-$C^1$}}

\newcommand{\grad}[1]{\nabla #1}
\newcommand{\gradp}[1]{\nabla_\perp #1}
\renewcommand{\div}[1]{\nabla \cdot #1}
\newcommand{\divp}[1]{\nabla_\perp \cdot #1}
\newcommand{\curl}[1]{\nabla \times #1}

\newcommand{\dotdot}{:}
\newcommand{\dottimes}{\dot\times}
\newcommand{\timestimes}{\stackrel{\times}{\times}}

\newcommand{\gs}[1]{\Delta^* #1}
\newcommand{\lp}[1]{\nabla^2 #1}
\newcommand{\pb}[2]{\left[#1,#2\right]}
\newcommand{\ip}[2]{\left\langle  #1,#2\right\rangle}
\newcommand{\funcss}[2]{
  \left\langle\left\langle #1,#2 \right\rangle\right\rangle}
\newcommand{\funcsa}[2]{\left[\left\langle #1,#2 \right\rangle\right]}
\newcommand{\funcaa}[2]{\left[\left[ #1,#2 \right]\right]}

\newcommand{\cola}[1]{\textcolor{red}{#1}}
\newcommand{\colb}[1]{\textcolor{blue}{#1}}

\newcommand{\uvec}[1]{\ensuremath{\vec{\hat{#1}}}}
\newcommand{\n}{\ensuremath{\uvec{n}}}


\newcommand{\repositoryloc}{portal.pppl.gov/p/tsc/C1/svn}
\newcommand{\svnurl}{http://subversion.tigris.org/}


\section{Boundary Conditions}

In all cases, $f = 0$ on the boundary, and therefore also $\uvec{t}
\cdot \grad{f} = 0$.  Some other boundary conditions that may be
specified are as follows:

\begin{description}
\item[No normal flow (\texttt{inonormalflow=1})] Holds $\uvec{n} \cdot
  \u$ constant.
\item[No poloidal flow (\texttt{inoslip\_pol=1})] Holds $\uvec{t} \cdot
  \u$ constant.
\item[No toroidal flow (\texttt{inoslip\_tor=1})] Holds $\uvec{\tor}
  \cdot \u$ constant.
\item[No normal current (\texttt{inocurrent\_norm=1})] Holds $\uvec{n} \cdot
  \j$ constant.
\item[No poloidal current (\texttt{inocurrent\_pol=1})] Holds $\uvec{t} \cdot
  \j$ constant.
\item[No toroidal current (\texttt{inocurrent\_tor=1})] Holds $\uvec{\tor}
  \cdot \j$ constant.
\end{description}

\begin{eqnarray}
  \uvec{n} \cdot \u & = & 
  -R \uvec{t}\cdot\grad{U}+ \frac{1}{R^2} \uvec{n} \cdot \grad{\chi}
  \\
  \uvec{t} \cdot \u & = & 
  R \uvec{n}\cdot\grad{U} + \frac{1}{R^2} \uvec{t} \cdot \grad{\chi}
  \\
  \uvec{\tor} \cdot \u & = & R \omega
\end{eqnarray}

\begin{eqnarray}
  \uvec{n} \cdot \B & = & 
  -\frac{1}{R} \uvec{t}\cdot\grad{\psi}
  - \frac{1}{R^2} \uvec{n} \cdot \grad{f_\tor}
  \\
  \uvec{t} \cdot \B & = & 
  \frac{1}{R} \uvec{n}\cdot\grad{\psi} 
  \\
  \uvec{\tor} \cdot \B & = & \frac{F}{R}
\end{eqnarray}

\begin{eqnarray}
  \uvec{n} \cdot \j & = & 
  -\frac{1}{R} \uvec{t} \cdot \grad{F}
  + \frac{1}{R^2} \uvec{n} \cdot \grad{\psi_\tor}
  \\
  \uvec{t} \cdot \j & = & 
  \frac{1}{R} \uvec{n}\cdot\grad{(F + f_{\tor \tor})} 
  + \frac{1}{R^2} \uvec{t} \cdot \grad{\psi_\tor}
  \\
  \uvec{\tor} \cdot \j & = & -\frac{1}{R}\gs{\psi}
\end{eqnarray}
In the above definitions, $\uvec{n}$ is the unit vector normal to the
boundary surface, and $\uvec{t} = \uvec{\tor} \times \uvec{n}$.

\subsection{Resistive Wall Boundary Condition}

For a thin resistive wall of resistivity $\eta_W$ and width
$\delta_W$, the following equations are obtained at the
boundary~\cite{Jardin10}:
\begin{eqnarray}
  \ddt{\psi} & = & -\frac{\eta_W}{\delta_W} \left(\uvec{n} \cdot
  \grad{\psi} - R \uvec{t} \cdot \B_v \right)
    \\
  \ddt{F} & = & -\frac{\eta_W}{\delta_W} \left(F - R \uvec{\tor} \cdot
  \B_v \right)
\end{eqnarray}
Here, $\B_v$ is the magnetic field on the outside of the wall.  $B_v$
may be obtained from the magnetic field on the inside of the wall
using a vacuum response matrix calculated by the $\textsc{vacuum}$
code,~\cite{Chance10} for example.

\subsubsection{Using a resistive wall in \codename\ with \textsc{vacuum}}

The following assumes that the \textsc{vacuum} code is located in the
directory \texttt{\$VACUUM\_DIR}.  On \texttt{sunfire},
\texttt{\$VACUUM\_DIR = /u/chance/Vacuum\_SVN\_Work}

\begin{enumerate}
\item Build \textsc{struct2vac} with\\
  \texttt{make OPT=1 struct2vac}
\item Generate \texttt{ordered.points} by running \textsc{struct2vac}
  in the working directory.
\item \texttt{tac ordered.points > vacin\_c1}
\item Edit \texttt{vacin\_c1}, and move the last line to be the first
  line.
\item \texttt{cp vacin\_c1 ordered.points}
\item \texttt{cp \$VACUUM\_DIR/Runs/C1Vac/modiv\_m3dc1 .}
\item \texttt{\$VACUUM\_DIR/Bin/vacuum.x86\_64.linux vacin\_c1 modiv\_m3dc1}
\item In \texttt{C1input}, set \texttt{eta\_wall} and
  \texttt{delta\_wall} to the appropriate values.
\end{enumerate}

Presently, the resistive wall only works on single-process runs.



\section{Discretization}

\subsection{Finite Elements}

Each field is represented as a linear combination of $N$ basis
functions $\nu_i$ on the computational domain
\[ U = \sum_{i=1}^N \nu_i U_i. \]
The finite element used in \codename is the reduced quintic element
\cite{Jardin04}, in which the basis functions are fifth order
polynomials.  At each time step, the projection of the equations onto
the basis functions are computed and solved.  For example, the
equation
\[ \ddt{U} = F(U) \]
becomes the system of projection equations
\[ \int dV\, \nu_i \ddt{U} = \int dV\, \nu_i F(U). \]
These projections equations are known collectively as the \emph{weak
form} of the equation.  Solving the equation in this manner is known
as the \emph{Galerkin method}.  Hereafter the index $i$ will be
dropped from $\nu_i$.

Once the equations are cast in the weak form, integrations by parts
may be carried out in order to reduce the order of the differential
operators acting on the physical fields.  For example, 
\begin{eqnarray*}
  \int dV\, \nu \nabla^2 U 
  & = & \int dV\, \div{(\nu \grad{U})} - \grad{\nu} \cdot \grad{U}\\
  & = & \oint d\vec{A} \cdot \grad{U} \nu - \int dV\, \ip{\nu}{\nabla U}\\
  & = & - \int dV\, \ip{\nu}{U}.
\end{eqnarray*}
It is found that using integrations by parts to re-cast the equations
into a form in which a roughly equal number of derivatives acts on the
trial function as on the physical fields improves the numerical
stability of methods for solving the equations.  Thus, in the above
example, the form $-\ip{\nu}{U}$ is preferable to $\nu \nabla^2
U$.

\subsubsection{Weak form of Physical Equations}

\texttt{Integration Identities}
\\
Rather than performing integrations by parts directly on each term in
equations~(\ref{eq:scalar_equations}), it is simpler to begin directly
from the vector form, equations~(\ref{eq:xmhd}) and use the following
identities when applying the operations to extract the scalar
equations:
\begin{eqnarray*}
  -\int dV\, R^2 \nu \grad{\tor} \cdot \curl{\vec{A}} & = & 
  -\int dV\, \vec{A} \cdot \left[\grad{(R^2 \nu)} \times \grad{\tor}
    \right]
  \\
  \int dV\, \nu \div{\vec{A}} & = & 
  -\int dV\, \grad{\nu} \cdot \vec{A}.
\end{eqnarray*}
(Note that the torodal operator, $R^2 \nu \grad{\tor} \cdot$, is not a
differential operator and therefore the integration by parts cannot be
performed \emph{a priori}.)  

Similary, useful identities for the operators that will act on the
stress tensor $\P$ are:
\begin{subequations}
\label{eq:tensor_identities}
\begin{eqnarray}
  R^2 \nu \grad{\tor} \cdot \curl{(\div{\P})} & = & 
  R^2 \partial_Z \nu \grad{\tor} \cdot \P \cdot \grad{\tor}
  - \grad{\nu} \cdot \P \cdot \grad{Z}
  \\ && \mbox{}
  + r \grad{\tor} \cdot \left[\grad{\grad{(\nu r)}} \dottimes \P\right]
  + \div{ \vec{A}_1 }\nonumber
  \\
  -R^2 \nu \grad{\tor} \cdot (\div{\P}) & = &
  R^2 \grad{\nu} \cdot \P \cdot \grad{\tor}
  + \div{\vec{A}_2}
  \\
  -\nu \div{(\div{\P})} & = & -\grad{\grad{\nu}} \dotdot \P + \div{\vec{A}_3}
\end{eqnarray}
\end{subequations}
where
\begin{eqnarray*} 
  \vec{A}_1 & = & 
  - R^2 \nu \grad{\tor} \times (\div{\P})
  - r \P \cdot \left[ \grad{\tor} \times \grad{(r \nu)} \right]
  + \nu \P \cdot \grad{Z}
  \\
  \vec{A}_2 & = & -R^2 \nu \P \cdot \grad{\tor}
  \\
  \vec{A}_3 & = & \grad{\nu} \cdot \P - \nu \div{\P}.
\end{eqnarray*}
(These identities hold for any symmetric tensor $\P$.)  The total
divergences vanish upon integration.

\subsubsection{Physical Equations after Integrations by Parts}

\begin{subequations}
  \label{eq:equations_ibp}
\begin{eqnarray}
  \int dV\, N_n & = & \int dV\, \left[
    N_{n U} + N_{n \chi} + N_{n D} \right]
  \\
  \int dV\, \left[U_{U n} + U_{\chi n}\right] & = & \int dV\, 
  \left[ 
    U_{U U n} + U_{V V n} + U_{U \chi n} + U_{\chi \chi n} 
    \right. \\ && \nonumber \left. \mbox{} 
    + U_{\psi \psi} + U_{F F} + U_{U \mu} + U_{\chi \mu} + U_g
    \right. \\ && \nonumber \left. \mbox{} 
    + U_{U D} + U_{\chi D} + U_{\P_\parallel} + U_{\P_\times}
    \right]
  \\
  \int dV\, V_{V n} & = & \int dV\, \left[
    V_{V U n} + V_{V \chi n} + V_{\psi F} + V_{V \mu} 
    \right. \\ && \nonumber \left. \mbox{} + V_{V D}  
    + V_{\P_\parallel} + V_{\P_\times} \right]
  \\
  \int dV\, \left[X_{U n} + X_{\chi n}\right] & = & \int dV\, \left[
    X_{U U n}+ X_{V V n}+ X_{U \chi n}+ X_{\chi \chi n} 
    \right.\\  && \nonumber \left. \mbox{} 
    + X_p + X_{\psi \psi} + X_{F F} + X_{U \mu} + X_{\chi \mu} + X_g 
    \right. \\ && \nonumber \left. \mbox{} 
    + X_{U D} + X_{\chi D} + X_{\P_\parallel} + X_{\P_\times}
    \right]
  \\
  \int dV\, \Psi_\psi & = & \int dV\, \left[
    \Psi_{\psi U} + \Psi_{\psi \chi} + \Psi_{\psi F n}
    + \Psi_{\psi \eta} \right]
  \\
  \int dV\, F_F & = & \int dV\, \left[
    F_{F U} + F_{\psi V} + F_{F \chi} + F_{\psi n} + F_{F n} 
    \right. \\ && \nonumber \left. \mbox{} + F_{p_e n} + F_{F \eta} \right]
  \\
  \int dV\, P_p & = & \int dV\, \left[
    P_{p U} + P_{p \chi} + P_{p_e F n} + P_{\eta \psi} + P_{\eta F} 
    \right. \\ && \nonumber \left. \mbox{} 
    + P_\kappa + P_{\kappa_\parallel} + P_{\kappa_\times} \right]
\end{eqnarray}
\end{subequations}

The terms in the above equations are categorized and defined in the
following sections.  Each term has been integrated by parts to arrive
at the simplest expression having for which the order of
differentiation on the trial function is roughly equal to that on the
physical fields.

\texttt{Basic Terms}
\\
The terms in this section are the basic terms in the two-fluid
equations, which do not depend on any specific choice of closure.

\begin{equation}
  \begin{array}{ll}
  N_n(\nu, \dot{n}) & = \nu \dot{n}\\
  N_{n U}(\nu, n, U) & = \nu \pb{U}{n}\\
  N_{n \chi}(\nu, n, \chi) & = n \ip{\nu}{\chi}\\
  N_{n D}(\nu, n, D) & = - D \ip{\nu}{n}
  \end{array}
\end{equation}

\begin{equation}
  \begin{array}{lcl}
    U_{U n}(\nu, \dot U, n) & = & -\frac{1}{R^2} n \ip{R^2 \nu}{\dot{U}}
    \\
    U_{\chi n}(\nu, \dot \chi, n) & = & -R^2 \nu \pb{n}{\dot{\chi}}
    \\
    U_{U U n}(\nu, U, U, n) & = & \frac{1}{R^2} n \gs{U} \pb{R^2\nu}{U}
      + \frac{1}{2 R^2} \ip{U}{U}\pb{R^2\nu}{n}
    \\
    U_{V V n}(\nu, V,  V, n) & = &  \frac{1}{2 R^2} \pb{\nu}{R^2} V V n
    \\
    U_{U \chi n}(\nu, U, \chi, n) & = & 
      \frac{1}{R^2}n \gs{U}\ip{R^2\nu}{\chi} 
      - \pb{U}{\chi} \pb{R^2\nu}{n}
    \\
    U_{\chi \chi n}(\nu, \chi, \chi, n) & = &
      \frac{1}{2} \ip{\chi}{\chi} \pb{R^2 \nu}{n}
    \\
    U_{\psi \psi}(\nu, \psi, \psi) & = &
      -\frac{1}{R^2} \pb{R^2 \nu}{\psi} \gs{\psi}
    \\
    U_{F F}(\nu, F, F) & = & -R^2 \nu F \pb{F}{\frac{1}{R^2}}
    \\
    U_{U D}(\nu, U, D) & = & \frac{1}{R^2} \ip{R^2 \nu}{U} D
    \\
    U_{\chi D}(\nu, \chi, D) & = & -\pb{R^2 \nu}{\chi} D
  \end{array}
\end{equation}

\begin{equation}
  \begin{array}{lcl}
    V_{V n}(\nu, V, n) & = & \nu n \dot{V}\\
    V_{V U n}(\nu, V, U, n) & = & \nu n \pb{U}{V}\\
    V_{V \chi n}(\nu, V, \chi, n) & = & -\nu n \ip{\chi}{V}\\
    V_{\psi F}(\nu, \psi, F) & = & \nu \pb{F}{\psi}\\
    V_{V D}(\nu, V, D) & = & -\nu V D
  \end{array}
\end{equation}    

\begin{equation}
  \begin{array}{lcl}
    X_{U n}(\nu, \dot U, n) & = & \nu \pb{n}{\dot{U}}
    \\
    X_{\chi n}(\nu, \dot \chi, n) & = & -n \ip{\nu}{\dot{\chi}}
    \\
    X_p(\nu, p) & = & \ip{\nu}{p}
    \\
    X_{U U n}(\nu, U, U, n) & = & -\frac{1}{R^2} n \gs{U} \ip{\nu}{U}
      + \frac{1}{2} n \ip{\nu}{\frac{\ip{U}{U}}{R^2}}
    \\
    X_{V V n}(\nu, V,  V, n) & = & 
      \frac{1}{2} n V V \ip{\frac{1}{R^2}}{\nu}
    \\
    X_{U \chi n}(\nu, U, \chi, n) & = & 
      \left( n \lp{\nu} + \ip{n}{\nu} \right) \pb{U}{\chi}
      + n \gs{U} \pb{\nu}{\chi}
    \\
    X_{\chi \chi n}(\nu, \chi, \chi, n) & = & \frac{1}{2} n 
      \ip{\nu}{\ip{\chi}{\chi}}
    \\
    X_{\psi \psi}(\nu, \psi, \psi) & = & 
      \frac{1}{R^2} \gs{\psi} \ip{\nu}{\psi}
    \\
    X_{F F}(\nu, F, F) & = & \frac{1}{R^2} F \ip{\nu}{F}
    \\
    X_{U D}(\nu, U, D) & = & \pb{\nu}{U} D
    \\
    X_{\chi D}(\nu, \chi, D) & = & \ip{\nu}{\chi} D
  \end{array}
\end{equation}


\begin{equation}
  \begin{array}{lcl}
    \Psi_{\psi}(\nu, \dot{\psi}) & = & \nu \dot{\psi}\\
    \Psi_{\psi U}(\nu, \psi, U) & = & \nu \pb{U}{\psi}\\
    \Psi_{\psi \chi}(\nu, \psi, \chi) & = & -\nu \ip{\chi}{\psi}\\
    \Psi_{\psi F n}(\nu, \psi, F, n) & = & d_i \nu \frac{1}{n} \pb{\psi}{F}\\
    \Psi_{\psi \eta}(\nu, \psi, \eta) & = & 
        -\frac{1}{R^2} \ip{\psi}{R^2 \nu \eta} 
  \end{array}
\end{equation}

\begin{equation}
  \begin{array}{lcl}
    F_F (\nu, \dot{F}) & = & \nu \dot{F}\\
    F_{F U}(\nu, F, U) & = & R^2 \nu \pb{U}{\frac{F}{R^2}}\\
    F_{\psi V}(\nu, \psi, V) & = & R^2 \nu \pb{\frac{V}{R^2}}{\psi}\\
    F_{F \chi}(\nu, F, \chi) & = & \frac{F}{R^2} \ip{R^2 \nu}{\chi}\\
    F_{\psi n}(\nu, \psi, \psi, n) & = &
       d_i \frac{\gs{\psi}}{R^2 n}\pb{\psi}{R^2\nu}\\
    F_{F n}(\nu, F, F, n) & = &
       d_i R^2 \nu F \pb{\frac{1}{R^2 n}}{F}\\
    F_{p_e n}(\nu, p_e, n) & = & d_i R^2 \nu \pb{\frac{1}{n}}{p_e}\\
    F_{F \eta}(\nu, F, \eta) & = & -\frac{1}{R^2} \eta \ip{R^2 \nu}{F}
  \end{array}
\end{equation}

\begin{equation}
  \begin{array}{lcl}
  P_p(\nu, \dot{p}) & = & \nu \dot{p}
  \\
  P_{p U}(\nu, p, U) & = & \nu \pb{U}{p}
  \\
  P_{p \chi}(\nu, p, \chi) & = & \Gamma p \ip{\nu}{\chi} 
    + (\Gamma - 1) \nu \ip{p}{\chi}
  \\
  P_{p_e, F, n}(\nu, p_e, F, n) & = & d_i \left( 
      \frac{1}{n} \nu \pb{p_e}{F} 
    + \Gamma \nu p_e \pb{\frac{1}{n}}{F} \right)
  \\
  P_{\eta, \psi}(\nu, \eta, \psi, \psi) & = & (\Gamma - 1) \nu
  \frac{(\gs{\psi})^2}{R^2}
  \\
  P_{\eta, F}(\nu, \eta, F, F) & = & (\Gamma - 1) \nu \frac{F^2}{R^2}
  \end{array}
\end{equation}

\texttt{Gravity}
\\
These terms are obtained assuming a gravitational force of the form given by
equation~(\ref{eq:gravity}).

\begin{equation}
  \begin{array}{lcl}
    U_g(\nu, n) & = & g_r \nu \pb{n}{R} - g_Z r \nu \ip{n}{R}
    \\
    X_g(\nu, n) & = & \frac{n}{R^2} \left( 
    g_r \ip{\nu}{R} + g_Z r \pb{\nu}{R} \right)
  \end{array}
\end{equation}



\texttt{Heat Flux Terms}
\\
These terms are obtained assuming a heat flux density of the form
described in section~\ref{sec:heat_flux}.

\begin{equation}
  \begin{array}{lcl}
    P_\kappa(\nu, \kappa, T) & = &
    -(\Gamma - 1) \kappa \ip{\nu}{T}
    \\
    P_{\kappa_\parallel}(\nu, \kappa_\parallel, T, \psi, \psi, B^{-2}) & = &
    -(\Gamma - 1) \kappa_\parallel \frac{1}{B^2} \pb{\psi}{\nu} \pb{\psi}{T}
    \\
    P_{\kappa_\times}(\nu, \kappa_\times, T, F, B^{-2}) & = & 
    (\Gamma - 1) \kappa_\times \frac{F}{B} \pb{\nu}{T}
  \end{array}
\end{equation}

\begin{eqnarray*}
  T & = & p/n \\
  B^2 & = & \frac{1}{R^2} \left[ \ip{\psi}{\psi} + F^2 \right]
\end{eqnarray*}


\texttt{Isotropic Viscosity}
\\
These terms result from isotropic viscosity of the form given by
equation~(\ref{eq:general_viscosity}).

\begin{equation}
  \begin{array}{lcl}
    U_{U \mu}(\nu, U, \mu) & = & \frac{1}{R^2} \left [ \left(
      \ip{\mu}{R^2 \nu} + \mu \gs{(R^2 \nu)} \right) \gs{U} \right. \\
      & & \left. \mbox{} + \lp{\mu} \ip{R^2 \nu}{U} 
      + \gs{(R^2 \nu)} \ip{\mu}{U} \right]
    \\
    U_{\chi \mu}(\nu, \chi, \mu) & = & -\lp{(R^2 \nu)} \pb{\mu}{\chi}
      - \gs{\mu}\pb{R^2\nu}{\chi} \\ & & \mbox{}
      - \frac{1}{R^2}\gs{(R^2 \chi)} \pb{R^2 \nu}{\mu}
    \\
    V_{V \mu}(\nu, V, \mu) & = & \left[\ip{\nu}{\mu} 
      + \frac{1}{R^2}\mu \gs{(R^2 \nu)} \right] V
    \\
    X_{U \mu}(\nu, U, \mu) & = & \lp{\nu} \pb{\mu}{U} 
      + \lp{\mu} \pb{\nu}{U} + \gs{U} \pb{\nu}{\mu}
    \\
    X_{\chi \mu}(\nu, \chi, \mu, \mu_c) & = & 
      \lp{\nu} \ip{\mu}{\chi} + \lp{\mu} \ip{\nu}{\chi}
      + 2 \mu_c \lp{\nu} \lp{\chi}
  \end{array}
\end{equation}

\texttt{Parallel Viscosity}
\\
These terms are obtained assuming a parallel viscosity of the form
given in equation~(\ref{eq:parallel_viscosity}).  These equations were
obtained using equations~(\ref{eq:tensor_identities}).  For
compactness, derivatives are written as subscripts in the following
expressions (\textit{i.e.} $\nu_Z = \partial_Z \nu$).

\begin{equation}
  \begin{array}{lcl}
    U_{\P_\parallel U}(\nu, U) & = & {\mu_\parallel}_U D_U
    \\
    U_{\P_\parallel V}(\nu, V) & = & {\mu_\parallel}_V D_U
    \\
    U_{\P_\parallel \chi}(\nu, \chi) & = & {\mu_\parallel}_\chi D_U
  \end{array}
\end{equation}

\begin{equation}
  \begin{array}{lcl}
    V_{\P_\parallel U}(\nu, U) & = & {\mu_\parallel}_U D_V
    \\
    V_{\P_\parallel V}(\nu, V) & = & {\mu_\parallel}_V D_V
    \\
    V_{\P_\parallel \chi}(\nu, \chi) & = & {\mu_\parallel}_\chi D_V
  \end{array}
\end{equation}

\begin{equation}
  \begin{array}{lcl}
    X_{\P_\parallel U}(\nu, U) & = & {\mu_\parallel}_U D_X
    \\
    X_{\P_\parallel V}(\nu, V) & = & {\mu_\parallel}_V D_X
    \\
    X_{\P_\parallel \chi}(\nu, \chi) & = & {\mu_\parallel}_\chi D_X
  \end{array}
\end{equation}

\begin{eqnarray*}
  D_U & = & \frac{3}{B^2} \left\{ 
  - \frac{1}{2}R^2\pb{\nu}{\frac{\ip{\psi}{\psi}}{R^2}}
  + \ip{\psi}{\pb{\nu}{\psi}} 
  - \frac{1}{R^2} F^2 \nu_Z 
  \right.\\ && \left. \mbox{}
  - \frac{2}{R^2}\left[ \nu_Z (\psi_Z^2 - \psi_R^2) + 2\nu_r \psi_r \psi_Z
    \right]
  \right\}
  \\
  D_V & = & - 3 \frac{F}{B^2} \pb{\nu}{\psi}
  \\
  D_X & = & - \lp{\nu} 
  \left(1 - \frac{3}{R^2}\frac{\ip{\psi}{\psi}}{B^2} \right)
  \\ &&  \mbox{}
  + \frac{3}{R^2 B^2} \left(
    \frac{1}{2}R^2\ip{\nu}{\frac{\ip{\psi}{\psi}}{R^2}}
    - \ip{\psi}{\ip{\nu}{\psi}} 
    + \frac{1}{R} F^2 \nu_r \right)
\end{eqnarray*}

\begin{eqnarray*}
  {\mu_\parallel}_U & = & \eta_0 \frac{p_i \tau_i}{R^2 B^2} \left( 
  - \frac{1}{2} R^2 \pb{U}{\frac{\ip{\psi}{\psi}}{R^2}}
  + \ip{\psi}{\pb{U}{\psi}}
  - \frac{1}{R^2} F^2 U_Z \right)
  \\
  {\mu_\parallel}_V & = & -\eta_0 p_i \tau_i \frac{F}{B^2} 
  \pb{\psi}{\frac{V}{R^2}}
  \\
  {\mu_\parallel}_\chi & = & \eta_0 \frac{p_i \tau_i}{R^2 B^2} \left(
    \frac{1}{2} R^2 \ip{\chi}{\frac{\ip{\psi}{\psi}}{R^2}}
  - \ip{\psi}{\ip{\chi}{\psi}}
  + \frac{1}{R} F^2 \chi_r 
  \right.\\ && \left. \mbox{}
  + \lp{\chi} \ip{\psi}{\psi}
  \right)
\end{eqnarray*}


\texttt{Gyroviscosity}
\\
These terms are obtained using equations~(\ref{eq:tensor_identities})
assuming a gyroviscosity of the form given by
equation~(\ref{eq:gyroviscosity}).

\begin{eqnarray*}
  & \lefteqn{U_{\P_\times U}(\nu, U) = -\frac{p_i F}{2 R^3 B^2}} & 
  \\ & & \mbox{} \times
    \left\{ \begin{array}{l} 
      \left(1+\frac{3}{2 R^2}\frac{\ip{\psi}{\psi}}{B^2}\right)
      \left[ \begin{array}{r@{}l}
	     & \left( [R^3 \nu_Z]_Z - [R^3 \nu_r]_r \right)
	       \left( \left[\frac{U_R}{R}\right]_Z 
	            + \left[\frac{U_Z}{R}\right]_r \right) 
             \\ \mbox{}
	   - & \left( [R^3 \nu_r]_Z + [R^3 \nu_Z]_r \right)
	       \left( \left[\frac{U_Z}{R}\right]_Z 
	            - \left[\frac{U_R}{R}\right]_r \right) 
	     \end{array} \right]
       \\ \mbox{}
      + \frac{9}{2 r B^2} 
      \\ \mbox{} \times \left[ 
	\begin{array}{r@{}l}
	  \left( \psi_Z^2 - \psi_R^2 \right) &
	  \left( r \nu_Z \left[ 
	    \left( \frac{U_Z}{R} \right)_Z -
	    \left( \frac{U_R}{R} \right)_r \right]
          -\frac{1}{R^3} U_Z \left[ 
	    (R^3 \nu_Z)_Z - (R^3 \nu_r)_r \right] \right)
	  \\ 
	  \mbox{} + 2 \psi_r \psi_Z &
          \left( r \nu_Z \left[ 
	    \left( \frac{U_R}{R} \right)_Z +
	    \left( \frac{U_Z}{R} \right)_r \right]
          -\frac{1}{R^3} U_Z \left[ 
	    (R^3 \nu_r)_Z + (R^3 \nu_Z)_r \right] \right)
	  \end{array} \right] 
    \end{array} \right\}
\end{eqnarray*}

\begin{eqnarray*}
  & \lefteqn{U_{\P_\times V}(\nu, V) = -\frac{p_i}{B^2}} &
  \\ && \mbox{} \times
  \left\{ \begin{array}{l}
      \frac{1}{4 R^2} \left(1 - \frac{3 F^2}{B^2 R^2} \right) 
      \\ \mbox{} \times
      \left( \ip{\frac{V}{R^2}}{R^4 \pb{\psi}{\nu}}
            -\ip{\psi}{R^4\pb{\nu}{\frac{V}{R^2}}}
	    +\frac{1}{R^2}\pb{\nu}{R^6\ip{\frac{V}{R^2}}{\psi}} \right)
      \\ \mbox{}
      -\frac{3}{4 B^2} \pb{\psi}{\frac{V}{R^2}}
      \\ \mbox{} \times
      \left( 2 \ip{\psi}{\ip{\psi}{\nu}}
           - R^2 \ip{\nu}{\frac{\ip{\psi}{\psi}}{R^2}}
	   - \gs{\nu} \ip{\psi}{\psi}
	   + 6 \psi_Z \pb{\nu}{\psi} \right)
      \\ \mbox{}
      +\frac{9 F^2}{2 B^2 R^2} \nu_Z \ip{\psi}{\frac{V}{R^2}}
    \end{array} \right\}
\end{eqnarray*}

\begin{eqnarray*}
  & \lefteqn{U_{\P_\times \chi}(\nu, \chi) = -\frac{p_i F}{2 R^3 B^2}} &
  \\ && \mbox{} \times 
  \left\{ \begin{array}{l}
    \left[ \left( \chi_{R R} - \chi_{Z Z} \right)
           \left( [R^3 \nu_r]_r - [R^3 \nu_Z]_Z \right)
       +   2 \chi_{R Z} 
           \left( [R^3 \nu_r]_Z + [R^3 \nu_Z]_r \right)
           \right]
    \\ \mbox{}
    + \frac{3}{R^2 B^2} \left[ \begin{array}{l} 
	\left( \gs{\chi}[\psi_Z^2 - \psi_R^2] 
        - \chi_{Z Z} \psi_Z^2 + \chi_{R R} \psi_R^2 \right)
	\left( [R^3 \nu_r]_r - [R^3 \nu_Z]_Z \right)
	\\ \mbox{} + 2 \chi_{R Z} 
	\left( \psi_Z^2 [R^3 \nu_r]_Z
	      +\psi_R^2 [R^3 \nu_Z]_r \right)
	\\ \mbox{} - 2 \psi_r \psi_Z 
	\left( \left[\chi_{Z Z} - \frac{1}{R}\chi_r \right] [R^3 \nu_r]_Z
	      +\left[\chi_{R R} - \frac{1}{R}\chi_r \right][R^3 \nu_Z]_r 
	      \right)
      \end{array} \right]
  \end{array} \right\}
\end{eqnarray*}

\begin{eqnarray*}
  & \lefteqn{V_{\P_\times U}(\nu, U) = \frac{p_i}{4 r B^2}} &
  \\ && \mbox{} \times 
  \left\{ \begin{array}{l}
    \left(1 - \frac{3}{R^2} \frac{F^2}{B^2} \right)
    \left( \ip{\psi}{R \pb{U}{\nu}}+\ip{\nu}{R \pb{U}{\psi}}
          -\frac{1}{R^3} \pb{U}{R^4 \ip{\nu}{\psi}}
	  +U_r \pb{\nu}{\psi} + \frac{2}{R} \psi_Z \ip{\nu}{U} \right)
    \\ \mbox{} + 
    \frac{3}{R B^2} \pb{\psi}{\nu}
    \left( 2\ip{\psi}{\ip{U}{\psi}} - \gs{U}\ip{\psi}{\psi} 
          - \frac{1}{R^2}\ip{U}{R^2\ip{\psi}{\psi}} 
	  +\pb{\psi}{R^2}\pb{\psi}{U} \right)
    \\ \mbox{} - 
    \frac{18}{R^2} \frac{F^2}{B^2} \pb{U}{R} \ip{\nu}{\psi}
  \end{array} \right\}
\end{eqnarray*}

\begin{eqnarray*}
  V_{\P_\times V}(\nu, V) & = & \frac{p_i F R^2}{4 B^2}
  \left(1 - \frac{3}{R^2} \frac{\ip{\psi}{\psi} - F^2}{B^2} \right)
    \pb{\nu}{\frac{V}{R^2}}
\end{eqnarray*}

\begin{eqnarray*}
  & \lefteqn{V_{\P_\times \chi}(\nu, \chi) = \frac{p_i}{B^2}} &
  \\ && \mbox{} \times 
  \left\{ \begin{array}{l}
    \left( \frac{1}{R^2}\ip{\chi}{R^2 \ip{\nu}{\psi}}
          -\ip{\nu}{\ip{\chi}{\psi}} - \ip{\psi}{\ip{\nu}{\chi}} \right)
    \\ \mbox{} + \frac{3}{2 r B^2} \pb{\psi}{\nu}
    \left( \ip{\psi}{R \pb{\chi}{\psi}} 
         - \frac{1}{2} r \pb{\chi}{\ip{\psi}{\psi}} \right)
    \\ \mbox{} + \frac{3}{4 R^2} \frac{F^2}{B^2}
    \left( \ip{\psi}{\ip{\chi}{\nu}} + \ip{\nu}{\ip{\chi}{\psi}}
          -\ip{\chi}{\ip{\nu}{\psi}} - 2 \gs{\chi}\ip{\nu}{\psi} \right)
  \end{array} \right\}
\end{eqnarray*}

\begin{eqnarray*}
  & \lefteqn{X_{\P_\times U}(\nu, U) = \frac{p_i F}{2 R^2 B^2}} &
  \\ && \mbox{} \times  
  \left\{ \begin{array}{l}
  \funcss{\nu}{U} - R^2 \funcaa{\nu}{U} 
  + \frac{1}{R} \left[ U_r (\nu_{Z Z} - \nu_{R R})
                     -2 U_Z \nu_{R Z} - \frac{1}{R} U_r \nu_r \right]
  \\ \mbox{}
  + \frac{3}{R B^2} \left[ \begin{array}{l}
    \left( \left[\frac{U_Z}{R} \right]_Z 
         - \left[\frac{U_R}{R} \right]_r \right)
    \left(\nu_{Z Z} \psi_R^2 - \nu_{R R} \psi_Z^2
         +\frac{1}{R} \nu_r [\psi_Z^2 - \psi_R^2] \right)
    \\ \mbox{}
    + 2\nu_{R Z} \left( 
        \left[ \frac{U_R}{R} \right]_Z \psi_R^2
      + \left[ \frac{U_Z}{R} \right]_r \psi_Z^2
      - \frac{1}{R^2} U_Z [\psi_Z^2 - \psi_R^2] \right)
    \\ \mbox{}
    - 2 \psi_r \psi_Z \left( \begin{array}{l}
        \left[ \frac{U_R}{R} \right]_Z \nu_{R R}
      + \left[ \frac{U_Z}{R} \right]_r \nu_{Z Z}
      - \frac{1}{R^2} U_Z [\nu_{Z Z} - \nu_{R R}] 
      \\ \mbox{}
      - \frac{1}{R} \nu_r \left[ 
	  \left( \frac{U_R}{R} \right)_Z
	+ \left( \frac{U_Z}{R} \right)_r \right] 
      \end{array} \right)
    \end{array} \right]
  \end{array} \right\}
\end{eqnarray*}

\begin{eqnarray*}
  & \lefteqn{X_{\P_\times V}(\nu, V) = \frac{p_i}{4 B^2}} &
  \\ && \mbox{} \times 
  \left\{ \begin{array}{l}
    \left(1 - \frac{3}{R^2} \frac{F^2}{B^2} \right) \left(
    \frac{1}{R^2} \ip{\nu}{R^2\ip{\frac{V}{R^2}}{\psi}}
    - \ip{\psi}{\ip{\frac{V}{R^2}}{\nu}}
    - \ip{\frac{V}{R^2}}{\ip{\psi}{\nu}} \right)
    \\ \mbox{} + \frac{6}{B^2} \pb{\psi}{\frac{V}{R^2}}
    \left(\frac{1}{R} \ip{\psi}{R \pb{\nu}{\psi}}
    - \frac{1}{2} \pb{\nu}{\ip{\psi}{\psi}} \right)
    \\ \mbox{} - 6 \frac{F^2}{B^2} \ip{\psi}{\frac{V}{R^2}}
    \left[ \left(\frac{\nu_Z}{R^2} \right)_Z
         + \left(\frac{\nu_R}{R^2} \right)_r \right]
  \end{array} \right\}
\end{eqnarray*}


\begin{eqnarray*}
  & \lefteqn{X_{\P_\times \chi}(\nu, \chi) = -\frac{p_i F}{B^2}} &
  \\ && \mbox{} \times 
  \left\{ \begin{array}{l}
    \left(1 + \frac{3}{2 R^2} \frac{\ip{\psi}{\psi}}{B^2} \right)
    \funcsa{\nu}{\chi}
    \\ \mbox{} + \frac{3}{2 B^2} \left[ \begin{array}{r@{}l}
      & \left(- \frac{1}{2} \pb{\chi}{\ip{\psi}{\psi}}
	      + \frac{1}{R} \ip{\psi}{R\pb{\chi}{\psi}} 
             \right)
	\left( \left[ \frac{\nu_R}{R^2} \right]_r
	     + \left[ \frac{\nu_Z}{R^2} \right]_Z \right)
    \\ \mbox{}
     - & \left(- \frac{1}{2} \pb{\nu}{\ip{\psi}{\psi}}
               + \frac{1}{R} \ip{\psi}{R\pb{\nu}{\psi}} 
              \right)
         \left( \left[ \frac{\chi_R}{R^2} \right]_r
	      + \left[ \frac{\chi_Z}{R^2} \right]_Z \right)
      \end{array} \right]
  \end{array} \right\}
\end{eqnarray*}

\subsubsection{Spatial Integration}

The integrals required to calculate the weak-form equations of the
Galerkin method are computed numerically using a 79-point Gaussian
quadrature.  That is, the value of each field is calculated at 79
points for each triangular element, and a weighted sum of these values
is computed to approximate the integral.
\[
  \int dA\ f(x) \simeq \sum_{i=1}^{79} w_i f(x_i),
\]
where the integrand $f(x)$ is restricted to a single element.  The
sampling points and weights appropriate for a equilateral triangle are
taken from ref.~\cite{Dunavant85}.

The coordinates of the sampling points are given in the ``natural
coordinates'' $(\alpha, \beta, \gamma)$ in ref.~\cite{Dunavant85}.
These coordinates may be converted to cartesian coordinates $(r, Z)$
for an equilateral triangle $e$ having vertices
\[
\left\{\left(-\frac{\sqrt{3}}{2},-\frac{1}{2}\right),
\left(\frac{\sqrt{3}}{2},-\frac{1}{2}\right), (0,1)\right\} 
\]
using the linear transformation
\[ 
\phi_{n \to e}(\alpha, \beta, \gamma) = 
\left(\frac{\sqrt{3}}{2}(\beta-\gamma), \frac{1}{2}(3 \alpha - 1)
\right).
\]
The weights must be multiplied by the Jacobian of this transformation,
\[
\mathcal{J}_{\phi_{n \to e}} = \frac{3 \sqrt{3}}{4}.
\]
To find the coordinates of the sampling points for a general triangle
$g$ having vertices $\{(-b,0), (a,0), (0,c)\}$, as in
ref.~\cite{Jardin04}, one may use the linear transformation
\begin{eqnarray*}
  \phi_{e \to g}(r,Z) & = & 
    \left(\frac{a+b}{\sqrt{3}} x + \frac{a-b}{3} (1-y), 
    \frac{c}{3}(2y+1) \right) \\
  \mathcal{J}_{\phi_{e \to g}} & = &  \frac{2 c}{3 \sqrt{3}} (a+b).
\end{eqnarray*}
The transformation from natural coordinates to cartesian coordinates
for a triangle having vertices $\{(-b,0), (a,0), (0,c)\}$ is therefore
\begin{eqnarray*}
  \phi_{n \to g}(r,Z) & = & 
  \left(\frac{1}{2} (a+b) (\beta-\gamma) +
  \frac{1}{2} (a-b)(1-\alpha), c \alpha \right) \\
  \mathcal{J}_{\phi_{n \to g}} & = & \frac{1}{2} (a+b) c.
\end{eqnarray*}

The 79-point quadrature gives the exact results for integrands which
are polynomials of degree 20 (or less).  In the case of quintic finite
elements, this means the integration is exact for terms involving
products of three fields or fewer, not including the degree-five
trial function $\nu$.  In cylindrical geometry, the presence factors
of $1/R$ will cause the quadrature not to be exact, as $1/R$ is not in
the form of a polynomial.  The weights $w_i$ must also be multiplied
by $R_i$ in cylindrical coordinates to account for the Jacobian of the
transformation from cartesian to cylindrical coordinates.


\subsection{Time Step}

\subsubsection{Implicit Time Advance}

For the implicit time advance, equations~(\ref{eq:equations_ibp}) are
evaulated at the $\theta$-advanced time (\emph{e.g.} $F(\psi) \to
F(\psi + \theta \dt \dot{\psi} + \cdots)$), linearized (\emph{i.e.}
$\order{\dt^2}$ and higher are dropped), and then discretized
temporally according to the chosen time integration method
(\emph{i.e.} $\dot{\psi} \to (\psi^{(n+1)} - \psi^{(n)})/\dt$).


\begin{eqnarray}
  \begin{pmatrix}
    \cola{S^v_{1 1}} & \cola{R^v_{1 1}} &
    \colb{S^v_{1 2}} & \colb{R^v_{1 2}} & 
          S^v_{1 3}  &        0         &
          R^v_{1 4}  &       R^v_{1 3}
    \\
    \cola{R^B_{1 1}} & \cola{S^B_{1 1}} &
    \colb{R^B_{1 2}} & \colb{S^B_{1 2}} & 
          R^B_{1 3}  &       S^B_{1 3}  &
              0      &        0
    \\
    \colb{S^v_{2 1}} & \colb{R^v_{2 1}} & 
    \colb{S^v_{2 2}} & \colb{R^v_{2 2}} & 
          S^v_{2 3}  &        0         &
	  R^v_{2 4}  &       R^v_{2 3}
    \\
    \colb{R^B_{2 1}} & \colb{S^B_{2 1}} &
    \colb{R^B_{2 2}} & \colb{S^B_{2 2}} & 
          R^B_{2 3}  &       S^B_{2 3}  &
              0      &        0
    \\
          S^v_{3 1}  &       R^v_{3 1}  &
          S^v_{3 2}  &       R^v_{3 2}  &
          S^v_{3 3}  &        0         &
	  R^v_{3 4}  &       R^v_{3 3}  
    \\
          R^B_{3 1}  &       S^v_{3 1}  &
          R^B_{3 2}  &       S^v_{3 2}  &
          R^B_{3 3}  &       S^v_{3 3}  &
              0      &        0
    \\
          R^n_{3 1}  &        0         &
          R^n_{3 2}  &        0         &
          R^n_{3 3}  &        0         &
          S^n        &        0
    \\
          R^p_{3 1}  &        0         &
          R^p_{3 2}  &        0         &
          R^p_{3 3}  &        0         &
              0      &       S^p        &
  \end{pmatrix}
  \begin{pmatrix}
    \cola{U}\\ \cola{\psi}\\ 
    \colb{V}\\ \colb{F}   \\
    \chi \\ p_e \\ 
    n \\ p
  \end{pmatrix}^{(n+1)} = \nonumber \\
  \begin{pmatrix}
    \cola{D^v_{1 1}} & \cola{Q^v_{1 1}} &
    \colb{D^v_{1 2}} & \colb{Q^v_{1 2}} & 
          D^v_{1 3}  &        0         &
          Q^v_{1 4}  &       Q^v_{1 3}
    \\
    \cola{Q^B_{1 1}} & \cola{D^B_{1 1}} &
    \colb{Q^B_{1 2}} & \colb{D^B_{1 2}} & 
          Q^B_{1 3}  &       D^B_{1 3}  &
              0      &        0
    \\
    \colb{D^v_{2 1}} & \colb{Q^v_{2 1}} & 
    \colb{D^v_{2 2}} & \colb{Q^v_{2 2}} & 
          D^v_{2 3}  &        0         &
	  Q^v_{2 4}  &       Q^v_{2 3}
    \\
    \colb{Q^B_{2 1}} & \colb{D^B_{2 1}} &
    \colb{Q^B_{2 2}} & \colb{D^B_{2 2}} & 
          Q^B_{2 3}  &       D^B_{2 3}  &
              0      &        0
    \\
          D^v_{3 1}  &       Q^v_{3 1}  &
          D^v_{3 2}  &       Q^v_{3 2}  &
          D^v_{3 3}  &        0         &
	  Q^v_{3 4}  &       Q^v_{3 3}  
    \\
          Q^B_{3 1}  &       D^v_{3 1}  &
          Q^B_{3 2}  &       D^v_{3 2}  &
          Q^B_{3 3}  &       D^v_{3 3}  &
              0      &        0
    \\
          Q^n_{3 1}  &        0         &
          Q^n_{3 2}  &        0         &
          Q^n_{3 3}  &        0         &
          D^n        &        0
    \\
          Q^p_{3 1}  &        0         &
          Q^p_{3 2}  &        0         &
          Q^p_{3 3}  &        0         &
              0      &       D^p        &
  \end{pmatrix}
  \begin{pmatrix}
    \cola{U}\\ \cola{\psi}\\ 
    \colb{V}\\ \colb{F}   \\
    \chi \\ p_e \\ 
    n \\ p
  \end{pmatrix}^{(n)} +   
  \begin{pmatrix}
    Q_1 \\ Q_2 \\ 
    Q_3 \\ Q_4 \\
    Q_5 \\ Q_6 \\ 
    Q_7 \\ Q_8
  \end{pmatrix}
\end{eqnarray}

\subsubsection{Split Time Step Method}



Time is advanced using a split time-step method in which the velocity
field is advanced first, then the density and total pressure fields
are advanced separately, and finally the magnetic field and electron
pressure are advanced together.  Though the velocity and magnetic
field are advanced separately, the Alfv\'en and magnetosonic waves are
treated implicitly by using
equations~(\ref{eq:scalar_p}--\ref{eq:scalar_F}) to calculate
analytically the advanced-time values of the pressure and magnetic
field for use in the velocity time step.



\begin{eqnarray}
  \label{eq:velocity_advance}
  \lefteqn{
  \begin{pmatrix}
    \cola{S^v_{1 1}} & \colb{S^v_{1 2}} & S^v_{1 3}\\
    \colb{S^v_{2 1}} & \colb{S^v_{2 2}} & S^v_{2 3}\\
          S^v_{3 1}  &       S^v_{3 2}  & S^v_{3 3}\\
  \end{pmatrix} 
  \begin{pmatrix}
    \cola{U}\\ \colb{V}\\ \chi
  \end{pmatrix}^{(n+1)}}\\
  & = & 
  \begin{pmatrix}
    \cola{D^v_{1 1}} & \colb{D^v_{1 2}} & D^v_{1 3}\\
    \colb{D^v_{2 1}} & \colb{D^v_{2 2}} & D^v_{2 3}\\
          D^v_{3 1}  &       D^v_{3 2}  & D^v_{3 3}\\
  \end{pmatrix} 
  \begin{pmatrix}
    \cola{U}\\ \colb{V}\\ \chi
  \end{pmatrix}^{(n)}
  + 
  \begin{pmatrix}
    \cola{Q^v_{1 1}} & \colb{Q^v_{1 2}} & Q^v_{1 3}\\
    \colb{Q^v_{2 1}} & \colb{Q^v_{2 2}} & Q^v_{2 3}\\
          Q^v_{3 1}  &       Q^v_{3 2}  & Q^v_{3 3}\\
  \end{pmatrix} 
  \begin{pmatrix}
    \cola{\psi}\\ \colb{F}\\ p
  \end{pmatrix}^{(n)} \nonumber
  \\ & & \mbox{} + 
  \begin{pmatrix}
    \cola{O^v_{1}}\\
    \colb{O^v_{2}}\\
          O^v_{3} \\
  \end{pmatrix} \nonumber
\end{eqnarray}

\begin{eqnarray}
  \label{eq:density_advance}
  S^n n^{(n+1)} & = & D^n n^{(n)} + 
  \begin{pmatrix} R^n_1 & R^n_2 & R^n_3\end{pmatrix}
  \begin{pmatrix}\cola{U}\\ \colb{V}\\ \chi\end{pmatrix}^{(n+1)}
  \\ & & \mbox{} + 
  \begin{pmatrix}Q^n_1    &   Q^n_2   & Q^n_3\end{pmatrix}
  \begin{pmatrix}\cola{U}\\ \colb{V}\\ \chi\end{pmatrix}^{(n)} \nonumber
\end{eqnarray}


\begin{eqnarray}
  \label{eq:pressure_advance}
  S^p p^{(n+1)} & = & D^p p^{(n)} + 
  \begin{pmatrix}R^p_1 & R^p_2 & R^p_3\end{pmatrix}
  \begin{pmatrix}\cola{U}\\ \colb{V}\\ \chi \end{pmatrix}^{(n+1)}
  \\ & & \mbox{} + 
  \begin{pmatrix}Q^p_1 & Q^p_2 & Q^p_3\end{pmatrix}
  \begin{pmatrix}\cola{U}\\ \colb{V}\\ \chi \end{pmatrix}^{(n)} \nonumber
\end{eqnarray}


\begin{eqnarray}
  \label{eq:field_advance}
  \lefteqn{
  \begin{pmatrix}
    \cola{S^B_{1 1}} & \colb{S^B_{1 2}} & S^B_{1 3}\\
    \colb{S^B_{2 1}} & \colb{S^B_{2 2}} & S^B_{2 3}\\
          S^B_{3 1}  &       S^B_{3 2}  & S^B_{3 3}\\
  \end{pmatrix} 
  \begin{pmatrix}
    \cola{\psi}\\ \colb{F}\\ p_e
  \end{pmatrix}^{(n+1)}}\\
  & = & 
  \begin{pmatrix}
    \cola{D^B_{1 1}} & \colb{D^B_{1 2}} & D^B_{1 3}\\
    \colb{D^B_{2 1}} & \colb{D^B_{2 2}} & D^B_{2 3}\\
          D^B_{3 1}  &       D^B_{3 2}  & D^B_{3 3}\\
  \end{pmatrix} 
  \begin{pmatrix}
    \cola{\psi}\\ \colb{F}\\ p_e
  \end{pmatrix}^{(n)} +
  \begin{pmatrix}
    \cola{R^B_{1 1}} & \colb{R^B_{1 2}} & R^B_{1 3}\\
    \colb{R^B_{2 1}} & \colb{R^B_{2 2}} & R^B_{2 3}\\
          R^B_{3 1}  &       R^B_{3 2}  & R^B_{3 3}\\
  \end{pmatrix} 
  \begin{pmatrix}
    \cola{U}\\ \colb{V}\\ \chi
  \end{pmatrix}^{(n+1)} \nonumber
  \\ & & \mbox{} +
  \begin{pmatrix}
    \cola{Q^B_{1 1}} & \colb{Q^B_{1 2}} & Q^B_{1 3}\\
    \colb{Q^B_{2 1}} & \colb{Q^B_{2 2}} & Q^B_{2 3}\\
          Q^B_{3 1}  &       Q^B_{3 2}  & Q^B_{3 3}\\
  \end{pmatrix} 
  \begin{pmatrix}
    \cola{U}\\ \colb{V}\\ \chi
  \end{pmatrix}^{(n)} +
  \begin{pmatrix}
    \cola{O^B_{1}}\\
    \colb{O^B_{2}}\\
          O^B_{3} \\
  \end{pmatrix} \nonumber
\end{eqnarray}

\texttt{Linear Calculations}
\\
Linear calculations may be performed by calculating each matrix once,
and recalculating the matrix-vector products each time step with the
updated vectors.  This method is very efficient because the $S$
matrices need only be inverted once, and in all subsequent time steps
the only matrix operations carried out are addition and matrix-vector
multiplication.  Non-linear simulations require all the matrices to be
recalculated each time step, and the $S$ matrices must be inverted
each time step.

\texttt{Implementation of Electron Pressure Equation}
\\
Because it is the electron pressure which appears in the generalized
Ohm's law, equation~(\ref{eq:ohm}), if the electron pressure equation
is retained, it is solved with the magnetic field as the third row in
equation~(\ref{eq:field_advance}) so as to keep the fast magnetosonic
wave implicit.  In this case, the full pressure is evolved
independently in equation~(\ref{eq:pressure_advance}).  If the
electron pressure equation is not included, the third row in
equation~(\ref{eq:field_advance}) is the total pressure equation, and
the electron pressure is assumed to remain always at a specific
fraction of the total pressure, which is determined by the initial
conditions.


\subsubsection{Crank-Nicholson}

The Crank-Nicholson time step is defined by the following discretization:
\begin{eqnarray*}
  \frac{\partial U}{\partial t} & \to & 
  \frac{U^{(n+1)} - U^{(n)}}{\dt}\\
  U & \to & \thimp U^{(n+1)} + (1-\thimp) U^{(n)}.
\end{eqnarray*}
By Taylor expanding about $U$,
\begin{eqnarray*}
  U^{(n+1)} & = & U + \thimp\,\dt\,\dot{U} + \frac{1}{2} \thimp^2 \dt^2
  \ddot{U} + \frac{1}{6} \thimp^3 \dt^3 \dddot{U}+ \cdots
  \\
  U^{(n)} & = & U + (\thimp-1) \dt\, \dot{U} + \frac{1}{2} (\thimp-1)^2 \dt^2
  \ddot{U} + \frac{1}{6} (\thimp-1)^3 \dt^3 \dddot{U} + \cdots
\end{eqnarray*}
the trunctation error of the time-derivative operator can be
calculated directly:
\begin{eqnarray*}
  \Delta_{CN}(\dt, \thimp) & = & \frac{U^{(n+1)} - U^{(n)}}{\dt} -
 \dot{U}
 \\ 
 & = & 
  \left( \thimp-\frac{1}{2} \right) \dt\,\ddot{U}
  + \frac{1}{2} \left(\thimp^2 - \thimp + \frac{1}{3} \right) \dt^2 \dddot{U}
  + \cdots.
\end{eqnarray*}
When $\thimp=1/2$, the time differencing is ``time-centered'' because
the two points involved in the time differencing are equidistant in
the logical time coordinate from the point at which the field itself
is evaluated.  In this case, the leading-order truncation error is
$\order{\dt^2}$:
\begin{equation}
  \Delta_{CN}(\dt, \thimp=1/2) = \frac{1}{24} \dt^2 \dddot{U} + \cdots.
\end{equation}

\subsubsection{BDF2}

The BDF2 time step is defined by the following discretization:
\begin{eqnarray*}
  \frac{\partial U}{\partial t} & \to & 
  \frac{3 U^{(n+1)} - 4 U^{(n)} + U^{(n-1)}}{2\,\dt}\\
  U & \to & U^{(n+1)}.
\end{eqnarray*}
Taylor expanding about $U$:
\begin{eqnarray*}
  U^{(n+1)} & = & U
  \\
  U^{(n)} & = & U - \dt\, \dot{U} + \frac{1}{2} \dt^2 \ddot{U} 
  - \frac{1}{6} \dt^3 \dddot{U} + \cdots
  \\
  U^{(n-1)} & = & U - 2\,\dt\, \dot{U} + 2\,\dt^2 \ddot{U} 
  - \frac{4}{3} \dt^3 \dddot{U} + \cdots
\end{eqnarray*}
and the truncation error is
\begin{equation}
  \Delta_{BDF2}(\dt) = 
  \frac{3 U^{(n+1)} - 4 U^{(n)} + U^{(n-1)}}{2\,\dt} - \dot{U} = 
  - \frac{1}{3} \dt^2 \dddot{U}
  + \cdots.
\end{equation}

%%%%%%%%%%%%%%%%%%%%%%%%%%%%%%%%%%%%%%%%%%
\section{Input Parameters}
\label{sec:input_parameters}
%%%%%%%%%%%%%%%%%%%%%%%%%%%%%%%%%%%%%%%%%%
\subsection{Model Options}

\begin{tabular}{llp{3in}}
  \textbf{Option}&\textbf{Default}&\textbf{Description}\\
  \hline
  \texttt{numvar} & 3   & MHD model. 1: 2-field;  2: 4-Field;  3: 6-Field.\\
  \texttt{linear} & 0   & 1: linear (perturbation terms only, no matrix
  recalculation)\\
  \texttt{eqsubtract}& 0& 1: remove equilibrium terms from equations\\
  \texttt{icsubtract}& 0& set to 1 if PF coils are in the domain.  These are
  defined in the files ``coils.dat'' and ``current.dat''\\
  \texttt{extsubtract} & 0 & 1: subtract fields from non-axisymmetric coils \\
  \texttt{idens}  & 1   & 1: include density equation\\
  \texttt{ipres}  & 0   & 1: include electron pressure equation\\
  \texttt{ipressplit} & 0 & 1: seperate pressure solve from the magnetic field
  solves when isplitstep=1.  (ipressplit must be 0 for isplitstep=0) \\
  \texttt{itemp} & 0 & 1: Advance temperatures rather than pressures (for ipressplit=1 only) \\
  \texttt{gyro}   & 0   & 1: include Braginskii gyroviscous term.  (note:
  needs db to be nonzero also) \\
  \texttt{igauge} & 0   & 0: loop voltage applied to boundary psi only \\
  \texttt{inertia} & 1  & 1: Include $\u \cdot \grad{\u}$ terms\\
  \texttt{itwofluid}& 1 & 1: Include $\j\times\B$ and
  $\grad{p_e}$ terms in Ohm's law (electron form).  2: ion form (not
  recommended)  3: parallel pressure gradient in Ohm's law only
  (not recommended) \\
  \texttt{ibootstrap} & 0 & 1: include bootstrap current \\
  \texttt{ibootstrap\_model} & 0 & 1: J\_BS = alpha F <p,psi> B \\
  \texttt{bootstrap\_alpha} & 0 & alpha parameter in bootstrap current model \\
  \texttt{imp\_bf} & 0 & 1: include implicit equation for f (recommended for
  3D and 2D complex) \\
  \texttt{nosig} & 0 & 1: drop sigma terms from momentum equation \\
  \texttt{itor}   & 0   & 0: cartesian; 1: cylindrical\\
  \texttt{istatic}& 0   & 1: Do not advance velocity\\
  \texttt{iestatic}&0   & 1: Do not advance magnetic fields\\
  \texttt{chiiner} & 1. & factor to multiply the chi equation inertial terms \\
  \texttt{ieq\_bdotgradt} & 1. & 1: include equilibrium parallel T gradient \\
  \texttt{no\_vdg\_T} & 0 & 1: do notinclude V dot grad T in Temp equation (debug) \\
  \texttt{iwall\_is\_limiter} & 1 & 1: wall acts as limiter \\
  \texttt{kinetic} & 0 & 1: Use kinetic PIC for hot pressure, 
                         2: Incompressible CGL,
                         3. Full CGL  
\end{tabular}

\begin{tabular}{llp{3in}}
  \textbf{Option}&\textbf{Default}&\textbf{Description}\\
  \hline
  \texttt{iadiabat} & 1 & 1: Corrects several problems with itemp=1 option \\
  \texttt{irunaway} & 0 & 1:  include runaway electron model \\
  \texttt{cre} & 0 & runaway speed \\
  \texttt{imp\_temp} & 0 & 0: compute temperatures for isplitstep=0, itemp=0 \\
  \texttt{iohmic\_heating} & 1 & 1: Include Ohmic heating term in heating \\
  \texttt{irad\_heating} & 1 & 1: Include radiation heat sink \\
  \texttt{gravr} & 0 & gravitational acceleration in R-direction \\
  \texttt{gravz} & 0 & gravitational acceleration in Z-direction 
\end{tabular}

\subsection{Initial Conditions Options}

\begin{tabular}{llp{4in}}
  \textbf{Option}&\textbf{Default}&\textbf{Description}\\
  \hline
  \texttt{itaylor} & 0 & \begin{minipage}[t]{2.5in}
    Pre-defined initial conditions.\\
 {\bf for itor=1 (toroidal geometry)} \\ 
    0: Tilting cylinder \\
    1: Calls Grad-Shafranov solver \\
    2: magneto-rotational equilibrium \\
    3: rotational instability \\   
    40: Fixed boundary stellarator \\
    41: Free boundary stellarator \\
 {\bf for itor=0 (slab geometry) } \\
    0: Tilting cylinder\\
    1: Taylor Reconnection\\
    2: Force-Free equilibrium (Taylor state) \\
    3: GEM Reconnection\\
    4: Wave Propagation\\
    5: Gravitational Instability\\
    6: Strauss equilibrium \\
    7: circular field init \\
    8,9:  biharmonic \\
    10,11,12,13:: analytic RWM test problem \\
    14: 3D wave test \\
    15: 3D diffusion test \\
    16:  FRS cylindrical equilibrium \\
    17:  ftz init \\
    18:  eigen init \\
    19:  ASDEX profiles similar to YU's \\
    20: kstar profiles with multiple q=1 surfaces \\
    21,22: fixed q(r) and p(r) profiles \\
    23:  Startsev equilibrium with $ J = (1/R_0q_0)(1 - r^2)$ \\
    27:  cylindrical test problem \\
    29:  basicJ profiles \\
  \end{minipage}\\
  \texttt{iupstream} & 0 & 1: addsdiffusion term to convection-like upstream differencing \\
  \texttt{magus}  & 5.e-2 & magnitude of the upstream diffusion term \\
  \texttt{iflip}    &  0 & 1: Flip coordinate system handedness\\
  \texttt{iflip\_b} &  0 & 1: Flip sign of toroidal field\\
  \texttt{iflip\_j} &  0 & 1: Flip sign of toroidal current\\
  \texttt{iflip\_v} &  0 & 1: Flip sign of toroidal velocity\\
  \texttt{iflip\_z} &  0 & 1: Flip equilibrium across z=0 plane \\
\end{tabular}

\begin{tabular}{llp{3in}}
  \textbf{Option}&\textbf{Default}&\textbf{Description}\\
  \hline
  \texttt{icsym}    &  0 &  
    \begin{minipage}[t]{2.5in}
    Symmetry of random perturbations \\
    0: No symmetry\\
    1: Odd up-down symmetry (in $U$)\\
    2: Even up-down symmetry (in $U$)
  \end{minipage}\\

  \texttt{bzero} & 1      & $B_\tor$ at \texttt{rzero}\\
  \texttt{bx0}  & 0 & Initial field in x-direction for some test problems \\
  \texttt{vzero} & 0 & Initial toroidal velocity for some test problems \\
  \texttt{phizero} & 0 & Initial velocity stream function for some test problems \\
  \texttt{v0\_cyl} & 0 & Central toroidal velocity for some test problems \\
  \texttt{v1\_cyl} & 0 & VZ=v0\_cyl + v1\_cyl*psin**beta \\
 \texttt{idevice}    &  0 &
    \begin{minipage}[t]{2.5in}
    define coils for a particular device \\
    -1: reads coil.dat file \\
    0: generic dipole configuration \\
    1: CDX-U \\
    2: NSTX \\
    3: ITER \\
    4: DIII-D
  \end{minipage}\\
  \texttt{iwave} & 0 & defines what wave to initialize in wave propagation test \\ 
  \texttt{eps}      &  0.01 & Size of random perturbation\\
  \texttt{maxn}     &  200 & Maximum wavenumber of initial random noise\\
  \texttt{verzero}  & 0 & magnitude of initial vertical velocity \\
   \texttt{irmp}    &  0 &
    \begin{minipage}[t]{2.5in}
    1: apply nonaxisymmetric fields throughout plasma.  
       reads rmp\_coil.dat for (R,Z) of window pane coils.  
       reads rmp\_current.dat for (+-) currents in kA and phases in degrees.  
       toroidal mode number of current specified by ntor  \\
    2: apply nonaxisymmetric fields only at aoundaries. 
  \end{minipage}\\

 \texttt{rmp\_atten}  &  0  & additional exponential decay of RMP field from r=1 for irmp=2 \\
 \texttt{iread\_ext\_field} & 0 & 1: read external field \\
 \texttt{beta}  & 0 & parameter used in some model equilibrium initializations \\
 \texttt{ln} & 0 & length scale parameter used in some model equilibrium \\
 \texttt{elongation} & 1 & elongation used in Solovev equilibrium

\end{tabular}

\begin{tabular}{llp{3in}}
  \textbf{Option}&\textbf{Default}&\textbf{Description}\\
  \hline

 \texttt{isample\_ext\_field} & 1 & factor to down sample external field data toroidally \\
 \texttt{isample\_ext\_field\_pol} & 1 & factor do down sample external field data poloidally \\
 \texttt{scale\_ext\_field} & 1 & factor to scale external field \\
 \texttt{shift\_ext\_field} & 0 & toroidal shift (in deg) of external fields \\
 \texttt{ibasicj\_solvep} & 0 & 0: uniform p, solve for F; 1: uniformF, solve for p \\
 \texttt{basicj\_nu} &1 & exponent in basicj equilibrium \\
 \texttt{basicj\_j0} & 1 & On-axis current density in basicj equilibrium \\
 \texttt{basicj\_voff} & 1 & Radial extent of flat toroidal rotation in basicj equilibrium \\
 \texttt{basicj\_vdelt} & 1 & Width of velocity drop-off, as fraction of ln, in basicj equilibrium \\
 \texttt{basicj\_dexp} & 1 & parameter for basicj equilibrium \\
 \texttt{basicj\_dvac} & 1 & parameter for basicj equilibrium \\
 \texttt{basicj\_q0} & 0 &   parameter for basicj equilibrium \\
 \texttt{basicj\_qa}  & 0 &   parameter for basicj equilibrium \\
 \texttt{pf\_shift} & 0 & (array) horizontal shift of PF coil \\
 \texttt{pf\_shift\_angle} & 0 & (array) direction of PF shift in degrees \\
 \texttt{pf\_tilt} & 0 & (array) Angle of PF from vertical in degrees \\
 \texttt{pf\_tilt\_angle} & 0 & (array) Axis of rotation for PF tilt in degrees \\
 \texttt{tf\_shift} & 0 & horizontal shift of TF coils \\
 \texttt{tf\_shift\_angle} & 0 & direction of TF shift in degrees \\
 \texttt{tf\_tilt} & 0 & angle of TF from vertical in degrees \\
 \texttt{tf\_tilt\_angle} & 0 & axis of rotation for TF tilt in degrees 

\end{tabular}

\subsection{Grad-Shafranov Solver Options}
\begin{tabular}{lcp{3in}}
  \textbf{Option}&\textbf{Default}&\textbf{Description}\\
  \hline
  \texttt{inumgs}& 0      & 1: Use numerical def. of p and g from profile-p and profile-g files\\
 \texttt{igs}   & 80     & Max number of Picard iterations\\
  \texttt{eta\_gs} & 1000.& factor for smoothing nonaxisymmetries in psi in 3D GS solve \\
  \texttt{igs\_pp\_ffp\_rescale} & 0 & 1: rescale p' and FF' to match p and F \\
  \texttt{nv1equ}& 0 & 1:use numvar =1 equilibrium for numvar .GT. 1 \\
  \texttt{tcuro} & 1	  & (scaled) plasma current in GS equilibrium\\
  \texttt{xmag}  & 1      & $R$-coordinate of magnetic axis\\
  \texttt{zmag}  & 0      & $Z$-coordinate of magnetic axis\\
  \texttt{xmag0} & 0      &  if nonzero, target magnetic axis $R$ for feedback\\
  \texttt{zmag0} & 0      &  if nonzero, target magnetic axis $Z$ for feecback\\
  \texttt{xlim}  & 0      & $R$-coordinate of limiter\\
  \texttt{zlim}  & 0      & $Z$-coordinate of limiter\\
  \texttt{xlim2}  & 0      & $R$-coordinate of limiter \#2\\
  \texttt{zlim2}  & 0	   & $Z$-coordinate of limiter \#2\\
  \texttt{rzero}  & 1      & nominal major radius of device for itor=1 \\
  \texttt{libetap}& 1.2    & approximate value of $l_i/2 + \beta_P$ for free-boundary equ \\
  \texttt{p0}    & 0.01   & Pressure at magnetic axis\\
  \texttt{pi0}   & 0.005  & Ion pressure at magnetic axis\\
  \texttt{p1}    & 0     & $p^{\prime}(\Psi)$ at magnetic axis\\
  \texttt{p2}    & 0     & $p^{\prime \prime}(\Psi)$ at magnetic axis\\
  \texttt{pedge} & 0	  & Pressure in vacuum region\\
  \texttt{tedge} & 0     & temperature in vacuum region (if .GT. 0).  Only
                           used in GS solve.   Boundary value of electron temp
                           is $twall = pedge \times pefac/den\_edge $ \\
 \texttt{tiedge} & 0     & ion temperature in vacuum region  \\
 \texttt{expn}  & 0 & \parbox[t]{3in}{Fraction of pressure gradient due to
    density gradient: $n = p^\mathtt{expn}$.}\\
 \texttt{q0}    & 1	  & Safety factor at magnetic axis\\
 \texttt{djdpsi}& 0	  & $J_\tor'(\Psi)$ at magnetic axis\\
 \texttt{th\_gs}& 0.8     & implicitness of GS Picard iterations\\
 \texttt{tol\_gs}& $10^{-8}$  & convergence criteria for GS iteration \\
  \texttt{pscale}  & 1.       & factor multiplying pressure profile \\
  \texttt{bscale}      &  1.0 & Factor multipying toroidal field\\
  \texttt{bpscale}     &  1.0 & Factor multiplying F' (keeping F0 constant) \\
  \texttt{vscale}      &  1.0 & Factor multiplying toroidal rotation profile \\
  \texttt{iread\_bscale}&  0   & 1: read profile\_bscale for factor to scale F \\
  \texttt{iread\_pscale} & 0   & 1: read profile\_pscale for factor to scale $p$ and $p^{\prime} $ \\
  \texttt{batemanscale} &  1   & Bateman scale the TF, keeping curent profile fixed \\

\end{tabular}

\begin{tabular}{llp{4in}}
  \textbf{Option}&\textbf{Default}&\textbf{Description}\\
  \hline
  \texttt{irot}         &  0   & 1: include toroidal rotation in equilibrium calculatin \\
  \texttt{iscale\_rot\_by\_p} & 1 & see below \\
  \texttt{alpha0}       &  0   & $\alpha_0$ in analytic rotation profile \\
  \texttt{alpha1}      	&  0   & $\alpha_1$ in analytic	rotation profile \\
  \texttt{alpha2}      	&  0   & $\alpha_2$ in analytic	rotation profile \\
  \texttt{alpha3}      	&  0   & $\alpha_3$ in analytic	rotation profile \\
\end{tabular}

For iread\_omega=0, the function $\alpha(\psi)$ is parameterized by:
 \[ \tilde{\alpha} = \alpha_{0} + \alpha_{1} s + \alpha_{2} s^{2} + \alpha_{3} s^{3} \]
For iscale\_rot\_by\_p = 0:  $\alpha = \tilde{\alpha} \times n(\psi) / p(\psi)$ . \\
For iscale\_rot\_by\_p = 1:  $\alpha = \tilde{\alpha} $. \\
For iscale\_rot\_by\_p = 2:  $\alpha = \left[ \alpha_{0} + \alpha_{1} e^{-\left[ \left( \psi - \alpha_{2} \right) / \alpha_{3} 
                                          \right]^{2} }       \right] \times n(\psi) /p(\psi) $ \\
In all cases, the angular velocity is then determined by:
\[      \omega = \left[  \frac{2 \alpha p(\psi)}{R_0^2 n(\psi)} \right]^{\frac{1}{2}} \]
\begin{tabular}{llp{6in}}
  \texttt{idenfunc}         & 0   &
  \begin{minipage}[t]{4.0in}
    0: $ n = \mbox{den0} \times (p/p0)^{\mbox{expn}} + \mbox{denedge} $ \\
    1: $ n = \mbox{den0} \times \frac{1}{2} \times 
       \left[1 + \tanh \left(\frac{\psi - (\psi_B + n_O (\psi_B - \psi_M))}
                                  {\Delta \times (\psi_B - \psi_M)        } \right)    \right] $ \\
    2: $ n = \mbox{den0} + \frac{1}{2}  \left( \mbox{den\_edge} - \mbox{den0} \right)
                  \times  \left[1 + \tanh \left( \frac{ \tilde{\psi} - n_O}
                                               {      \Delta            } \right)  \right] $\\
    3: if $\tilde{\psi}$ .LT. $n_O$ and $(\psi - \psi_M) \times \left[d \psi /dx (x - x_{MA}) + d \psi /dz (z - z_{MA} )           
                \right] $ .GT. 0, then $n$ = den0.     Else, $n$ = den\_edge.  \\
    ( $\psi_B = \mbox{psibound}, \psi_M = \mbox{psimin}, \tilde{\psi} = (\psi - \psi_M)/(\psi_B - \psi_M) $ )
  \end{minipage}    \\

  \texttt{den\_edge}        & 0.0 & edge density.  If 0, set to den0*(pedge/p0)**expn \\
  \texttt{den0}             & 1.0 & (scaled) central density\\
  \texttt{denoff}           & 1.0 & $n_O$: offset for idenfunc=1,2,3 \\
  \texttt{dendelt}          & 0.1 & $\Delta$: width of transition region for idenfunc=1,2 \\

  \texttt{divertors} & 0  & Number of divertors (0--2)\\
  \texttt{divcur}& 0.1    & Divertor current(s), as fraction of tcuro\\
  \texttt{xdiv}  & 0      & $r$-coordinate of divertor current(s)\\
  \texttt{zdiv}  & 0      & \parbox[t]{3in}{$z$-coordinate of 
    divertor 
    current.  If $\mathtt{divertors} = 2$, the second divertor has 
    $z = -\mathtt{zdiv}$.}\\
  \texttt{xnull}     & 0 & Guess for $r$-coordinate of x-point\\
  \texttt{znull}     & 0 & Guess for $z$-coordinate of x-point\\
  \texttt{mod\_null\_rs} & 0 & if 1: you can reset xnull and znull from C1input \\
  \texttt{xnull0}  &  0  & Target R-Coordinate of x-point for feedback \\
  \texttt{znull0}  &  0  & Target Z-Coordinate of x-point for feedback \\
  \texttt{xnull2}     & 0 & Guess for $r$-coordinate of inactive x-point\\
  \texttt{znull2}     & 0 & Guess for $z$-coordinate of inactive x-point\\
  \texttt{mod\_null\_rs2} & 0 & if1: you can reset xnull2 and znull2 from C1input \\
  \end{tabular}

  \begin{tabular}{llp{4.0in}}
  \textbf{Option}&\textbf{Default}&\textbf{Description}\\
  \hline
 \texttt{gs\_pf\_psi\_width}            & 0 & width of psi smoothing into provate flux region \\
 \texttt{gs\_vertical\_feedback}        & 0 & proportional feedback of each coil to (zmag-zmag0) (array) \\ 
 \texttt{gs\_vertical\_feedback\_i}     & 0 & integral feedback of each coil to (zmag-zmag0) (array) \\
 \texttt{gs\_vertical\_feedback\_x}     & 0 & proportional feedback of each coil to (znull-znull0) (array) \\
 \texttt{gs\_vertical\_feedback\_x\_i} & 0 & integral feedback of each coil to (znull-znull0) (array) \\
 \texttt{gs\_radial\_feedback}          & 0 & proportional feedback of each coil to (xmag-xmag0) (array) \\
 \texttt{gs\_radial\_feedback\_i}       & 0 & integral feedback of each coil to (xmag-xmag0) (array) \\
 \texttt{gs\_radial\_feedback\_x}       & 0 & proportional feedback of each coil to (xnull-xnull0) (array) \\
 \texttt{gs\_radial\_feedback\_x\_i}    & 0 & integral feedback of each coil to (xnull-xnull0) (array) \\
 \texttt{igs\_extend\_p}                & 0 & 1: extend p past pls=1 using ne and Te profiles  \\
 \texttt{igs\_feedfac}                  & 1 & proportionality factor for external field feedback \\
 \texttt{igs\_forcefree\_lcfs}          & -1 & 1: ensure that GS solution is force free at LCFS \\
 \texttt{igs\_start\_xpoint\_search}    & 0 &  number of GS iterations before searching for x-point \\
 \texttt{sigma0}                        & 0 &  width of Gaussian for initial current distribution for GS iteration \\
 \texttt{igs\_extend\_diagmag}          & 1 &  1: extend diamagnetic rotation past psi=1 \\
 
\texttt{adapt\_qs} & 0 & Safety factor values to pack around (array) \\
\texttt{adapt\_zlow} & 0 & Z-coordinate below which SOL adaption is coarse \\
\texttt{adapt\_zup}  & 0 & Z-coordinate above which SOL adaptation is coarse


\end{tabular}




\subsection{Transport Coefficients}

\begin{tabular}{llp{4.0in}}
  \textbf{Option}&\textbf{Default}&\textbf{Description}\\
  \hline
  \texttt{ivisfunc} & 0 & select viscosity function \\
                    &   & 0: $ \mbox{visc} = \mbox{amu} $ \\
                    &   & 1: $ \mbox{visc} = \mbox{amu} + \frac{1}{2} \mbox{amu\_edge} \times
                 \left[ 1. + \tanh \left[   \frac{\psi - \left( \psi_l+\nu_0(\psi_l - \psi_0)\right)}
                                                 {\nu_{\Delta} (\psi_l - \psi_0)                   }\right] \right] $  \\
                    &   & 2:  $ \mbox{visc} = \mbox{amu} + \frac{1}{2} \mbox{amu\_edge} \times
                 \left[ 1. + \tanh \left[   \frac{\tilde{\psi} - \nu_0}
                                                 {\nu_{\Delta}}  \right] \right] $  \\
                    &   & or, if amuoff2 .ne. 0 and amudelt2.ne.0) \\ 
                    &   &    $ \mbox{visc} = \mbox{amu} + \frac{1}{4} \mbox{amu\_edge} \times
                     \left[ 2. + \tanh \left[   \frac{\tilde{\psi} - \nu_0}
                                                 {\nu_{\Delta}}  \right]                     
                               + \tanh \left[   \frac{\tilde{\psi} - \nu_{02} }
                                                 {\nu_{\Delta2}}  \right]    \right] $  \\
                    &   & 3:  visc = amu or amu\_edge depending on criteria in define\_fields \\

  \texttt{amu}       & 0 & core viscosity for ivisfunc =0,..,3 \\
  \texttt{amu\_edge} & 0 & edge viscosity for ivisfunc = 1,..,3 \\
  \texttt{amuoff}    & 0 & $\nu_0$ in ivisfunc = 1,2 \\
  \texttt{amuoff2}   &   & $\nu_{02}$ in ivisfunc = 1,2 \\
  \texttt{amudelt}   & 0 & $\nu_{\Delta}$ in ivisfunc = 1,2 \\
  \texttt{amudelt2}  & 0 & $\nu_{\Delta2}$ in ivisfunc = 1,2 \\
  \texttt{amuc}   & 0 & Compressional viscosity coefficient\\
  \texttt{amupar} & 0 & Parallel viscosity coefficient \\
  \texttt{amue}  & 0 & bootstrap viscosity coefficient \\
  \hline
  \texttt{iresfunc} & 0 & select resistivity function  \\
                    &   & 0: eta = etar + eta0/Te**(3/2) \\
                    &   & 1: $ \mbox{eta} = \mbox{etar} + \frac{1}{2} \mbox{eta0} \times
                 \left[ 1. + \tanh \left[   \frac{\psi - \left( \psi_l+\mbox{etaoff} \times (\psi_l - \psi_0)\right)}
                                                 {\mbox{etadelt}\times (\psi_l - \psi_0)                   }\right] \right] $  \\
                    &   & 2:  $ \mbox{eta} = \mbox{etar} + \frac{1}{2} \mbox{eta0} \times
                 \left[ 1. + \tanh \left[   \frac{\tilde{\psi} - \mbox{etaoff}}
                                                 {\mbox{etadelt}}  \right] \right] $  \\
                    &   & The following two options are applied in a way that they \\ 
                    &   & should not have negative values...even if the idl plots \\
                    &   & indicate otherwise \\
                    &   & 3: eta = etar for $\tilde{\psi} < \mbox{etaoff}$ othrwise eta0  \\
                    &   & 4:  Spitzer resistivity with offset. \\
                    &   &     Define $T_{wall}$ = pedge*pefac/den\_edge \\
                    &   & $\mbox{for} T_e > T_{wall} - T_e^{off}, \eta = (T_e-T_e^{off})^{-3/2} $ \\
                    &   & $\mbox{for} T_e < T_{wall} - T_e^{off}, \eta = (T_{wall}-T_e^{off})^{-3/2} $ \\
                    &   & can be increased by inputing eta\_fac $>$ 1.  \\
                    &   & 5:  Simple neoclassical model:  \\
                    &   & $\eta = \mbox{eta0} \times (n_e/p_e)^{3/2} / (1. - 1.46 (r/R)^{1/2})   $ \\
 \texttt{etar}    & 0 & see description of iresfunc \\
  \texttt{eta0}   & 0 & see description of iresfunc \\
  \texttt{etaoff} & 0 & see description of iresfunc \\
  \texttt{etadelt} & 0 & see description of iresfunc \\
  \texttt{eta\_te\_offset} & 0 & $T_e^{off}$  for iresfunc=4 \\
  \texttt{ikprad\_te\_offset} & 0 & if 1, $T_e^{off}$ also used in kprad and ablation \\
  \texttt{eta\_fac} & 0 & for iresfunc=4, resistivity multiplied by eta\_fac \\
  \texttt{eta\_mod} & 0 & if 1: remove d/dphi terms in resistivity \\
  \texttt{eta\_max} & 0 & maximum resistivity in plasma (defaults to etavac) \\
  \texttt{eta\_min} & 0 & minimum resistivity in plasma  \\
\end{tabular}

\begin{tabular}{llp{4.0in}}
  \textbf{Option}&\textbf{Default}&\textbf{Description}\\
  \hline
  \texttt{ikappafunc} & 0 &  select thermal conductivity function \\
    & & 0: $\kappa = \mbox{kappat} + \mbox{kappa0} \times * (n^3/p)^{1/2}  $\\
    & & 1: $\kappa = \mbox{kappa0} \times \frac{1}{2}
                    \left[1 + \tanh \left[\frac{\psi -\left( \psi_l + \kappa^{0ff} \times (\psi_l - \psi_0)\right)}
                                               {\kappa_{\Delta} \times (\psi_l - \psi_0)} \right]  \right]$ \\
    & & 2: $\kappa = \mbox{kappa0} \times \frac{1}{2}
                    \left[  1 + \tanh \frac{\tilde{\psi} - \kappa^{off}}
                                           {  \kappa_{\Delta}}  \right] \mbox{for} \tilde{\psi} < 1 $  \\
    & & 2: $\kappa = \mbox{kappa0} \times \frac{1}{2}
                    \left[  1 + \tanh \frac{2 - \tilde{\psi} - \kappa^{off}}
                                           {  \kappa_{\Delta}}  \right] \mbox{for} \tilde{\psi} > 1 $  \\
    & & 3:  $\kappa = \mbox{kappat} + \mbox{kappa0} \times 1/(pn)^{1/2}$ \\
    & & 4:  $\kappa = \mbox{kappat} + \mbox{kappa0} \times ( 1. + \mbox{kappadelt} \times |\nabla T_e|^2 ) $  \\
    & & 5:  $\kappa = \mbox{kappat} + \mbox{kappa0}/T_e$ limited by kappa\_max  \\ 
    & & 10: read from profile\_kappa file in $m^2/\mbox{sec}$  \\ 
    & & 11: read from profile\_kappa file in normalized units  \\
    & & 12: option to go with itaylor=27   \\
 \texttt{kappa\_max}  & 0 & if .NE. 0, max $\kappa$ for ikappafunc=5 \\
 \texttt{kappai\_fac} & 1 & ion thermal conduction is kappai\_fac* kappa \\
 \texttt{ikapscale}   & 0 & if 1: kappar gets scaled by kappa  \\
 \texttt{ikappar\_ni} & 0 & 1: include 1/n terms in parallel heat flux \\
 \texttt{kappaoff}    & 0 & $\kappa^{off}$ see ikappafunc \\
 \texttt{kappadelt}   & 0 & $\kappa_{\Delta} $ see ikappafunc \\
 \texttt{kappat}      & 0 & isotropic thermal conductivity \\
 \texttt{kappa0}      & 0 & see ikappafunc \\
 \texttt{ikapparfunc} & 0 &  select parallel thermal conductivity (PTC) function \\
                      &   & 0: PTC = kappar  \\
                      &   & 1: PTC = $\mbox{kappar}/  \left[ (T_{crit}/T)^{5/2} + 1   \right] $ \\
 \texttt{kappar} & 0 & Parallel thermal conductivity\\
 \texttt{tcrit}  & 0 & $T_{crit}$ for ikapparfunc = 1  \\

 \texttt{kappari\_fac}& 1 & ion parallel thermal conductivity is kappari\_fac x electron value \\
 \texttt{kappax}      & 0 & coefficient of $B \times \nabla T$ temperature diffusion \\
 \texttt{kappah}      & 0 & if nonzero, $\mbox{kappa} = \mbox{kappah} \times \tanh^2 \left[ (\tilde{\psi}-1.)/2  \right] $ \\
  \texttt{kappaf} & 1 & Factor multiplying kappa when $\nabla p < \nabla p_{crit} $ \\
  \texttt{kappag} & 0 & Thermal diffusion proportional to pressure gradient \\
  \texttt{gradp\_crit} & 0 & $\nabla p_{crit} $ for kappaf,kappag model \\
  \texttt{k\_fac} & 1 & Factor by which TF is multiplied in denominator of kappa\_par \\
  \texttt{temin\_q0}  & 0 & Min temperature used in equipartition for ipres=1 \\
  \hline
  \texttt{idenmfunc}   & 0 & selects from of particle diffusion (PD) \\
                      &   &  0: PD = denm \\
                      &   &  1: PD = denm + denmt/Te \\
                      &   &  10: read from file profile\_denm in $m^2/\mbox{sec}$  \\
                      &   &  11: read from file profile\_denm in normalized units \\
  \texttt{denm}   & 0 &   see idenmfunc \\
  \texttt{denmt}  & 0 &  multiplier of 1/Te for idenmfunc=1 \\
  \texttt{denmmin} & 0 & minimum value of denm \\
  \texttt{denmmax} & 1.E6 & maximum value of denm \\
 
\end{tabular}

\subsection{Hyper-Diffusivity}

\begin{tabular}{lcp{2.2in}}
  \textbf{Option} & \textbf{Default} & \textbf{Description}\\
  \hline
  \texttt{xzero}  & 0 & $R$-coordinate of bottom left corner of domain\\
  \texttt{zzero}  & 0 & $Z$-coordinate of bottom left corder of domain\\
  \texttt{iper}   & 0 & 1: Left/right boundaries periodic\\
  \texttt{jper}   & 0 & 2: Top/bottom boundaries periodic\\
  \texttt{inonormalflow}& 1 & 1: No-normal-flow boundary\\
  \texttt{inoslip\_pol} & 0 & 1: No-slip boundaries for poloidal velocity\\
  \texttt{inoslip\_tor} & 1 & 1: No-slip boundaries for toroidal velocity\\
  \texttt{inostress\_tor}&0 & 1: No-normal-stress boundary for toroidal 
                                 velocity\\
  \texttt{com\_bc}& 1 & 1: $\nabla^2 \chi = 0$\\
  \texttt{iconst\_t}  & 0 & 0: $\partial_t T = 0$\\
  \texttt{inograd\_t} & 0 & 0: No normal temperature gradient\\
  \texttt{vloop}  & 0 & Loop voltage.  $\partial_t \psi = \mathtt{vloop}$\\
  \texttt{imask}  & 0 & 1: Smoothly bring $d_i$ to zero near
    boundaries\\
  \texttt{amu\_edge} & 0 &    Factor by which to increase viscosity near 
                              boundaries
\end{tabular}


\subsection{Time Integration Options}
\begin{tabular}{lcp{3in}}
  \textbf{Option}&\textbf{Default}&\textbf{Description}\\
  \hline
  \texttt{dt}         & 0.1 & Time step\\
  \texttt{ntimemax}   & 20  & Total number of time steps\\
  \texttt{integrator} & 0   & 0: Crank-Nicholson (CN); 1: BDF2\\
  \texttt{imp\_mod}   & 0   & 
  \begin{minipage}[t]{3in}
    0: $\theta$-implicit\\
    1: Implicit leapfrof (\texttt{isplitstep} = 1 only)\\
  \end{minipage}\\
  \texttt{thimp}      & 0.5 & Implicitness parameter\\
  \texttt{thimp\_ohm} & \texttt{thimp} & 
                              Implicitness of ohmic heating terms\\
  \texttt{thimpsm}    & 1   & Implicitness of the smoother functions\\
  \texttt{isplitstep} & 1   & 0: Fully implicit time step; 
                              1: split time step.\\
  \texttt{iresolve}   & 0   & 1: Re-solve velocity after 
                              field advance of split time step.\\
  \texttt{iteratephi} & 0   & 1: Calculate transport coefficients after
    field advance, then recalculate field advance.
\end{tabular}


\subsection{Spatial Integration Options}
\begin{tabular}{lcp{3in}}
  \textbf{Option}&\textbf{Default}&\textbf{Description}\\
  \hline
  \texttt{int\_pts\_main}  & 79 & Sampling points for integrations in
                                main time step matrices\\
  \texttt{int\_pts\_aux}   & 79 & Sampling points for integrations in
                                calculations of auxiliary variables\\
  \texttt{int\_pts\_diag}  & 79 & Sampling points for integrations in
                                diagnostic calculations\\
\end{tabular}


\subsection{Numerical Options}
\begin{tabular}{llp{3in}}
  \textbf{Option}&\textbf{Default}&\textbf{Description}\\
  \hline
  \texttt{ivform} & 0   & 0: $\u = \grad{U}\times\grad{\tor} + V
   \grad{\tor} + \grad{\chi}$\\
   & & 1: $\u = \r^2 \grad{U}\times\grad{\tor} + \r^2 \omega
   \grad{\tor} + \r^{-2} \grad{\chi}$\\
  \texttt{jadv}   & 0   & 1: Use toroidal current density equation
                          instead of poloidal flux equation.\\
  \texttt{max\_ke}& 1.0  & Maximum value of kinetic energy before solution is
                          rescaled in linear simulations. (0 = don't rescale)\\
  \texttt{harned\_mikic} & 0.0 & Coefficient of Harned-Mikic two-fluid
                          stabilization term.\\
\end{tabular}

\subsection{Input/Output Options}

\begin{tabular}{lcp{2.7in}}
  \textbf{Option}&\textbf{Default}&\textbf{Description}\\
  \hline
  \texttt{ntimepr}   & 5 & Number of timesteps per full field output\\
  \texttt{iprint}    & 0 & 1: Print detailed info to stdout\\
  \texttt{iglobalout}& 0 & 1: Write process-independent output file\\
  \texttt{iglobalin} & 0 & 1: Read process-independent output file on restart\\
  \texttt{irestart}  & 0 & 1: Read restart file\\
  \texttt{iread\_eqdsk}   & 0 & 1: Read EFIT g-file 'geqdsk'\\
  \texttt{iread\_dskbal}  & 0 & 1: Read BAL file 'dskbal'\\
  \texttt{iread\_jsolver} & 0 & 1: Read Jsolver file 'fixed'
\end{tabular}



\subsection{Boundary and Domain Options}

\begin{tabular}{lcp{2.5in}}
  \textbf{Option} & \textbf{Default} & \textbf{Description}\\
  \hline
  \texttt{xzero}  & 0 & $r$-coordinate of bottom left corner of domain\\
  \texttt{zzero}  & 0 & $z$-coordinate of bottom left corder of domain\\
  \texttt{iper}   & 0 & 1: Left/right boundaries periodic\\
  \texttt{jper}   & 0 & 2: Top/bottom boundaries periodic\\
  \hline
  \texttt{ifixedb} & 0 & Set $\psi=0$ on boundary\\
  \texttt{inonormalflow}& 1 & 1: No-normal-flow boundary\\
  \texttt{inoslip\_pol} & 0 & 1: No-slip boundaries for poloidal velocity\\
  \texttt{inoslip\_tor} & 1 & 1: No-slip boundaries for toroidal velocity\\
  \texttt{inostress\_tor}&0 & 1: No-normal-stress boundary for toroidal 
                                 velocity\\
  \texttt{iconst\_bz} & 1 & 1: Toroidal field held constant on boundary\\
  \texttt{iconst\_n}  & 0 & 1: Density held constant on boundary\\
  \texttt{iconst\_p}  & 1 & 1: Pressure held constant on boundary\\
  \texttt{iconst\_t}  & 0 & 1: Temperature held constant on boundary\\
  \texttt{inograd\_p} & 0 & 1: No normal pressure gradient\\
  \texttt{inograd\_t} & 0 & 1: No normal temperature gradient\\
  \texttt{com\_bc}& 0 & 1: $\nabla^2 \chi = 0$\\
  \texttt{vor\_bc}& 0 & 1: $\Delta^* U = 0$\\
  \texttt{imask}  & 0 & 1: Smoothly bring $d_i$ to zero near
    boundaries\\
  \hline
  \texttt{eta\_wall}   & 0 & Resistivity of the wall\\
  \texttt{delta\_wall} & 1 & Thickness of the resistive wall
\end{tabular}



\subsection{Unit Normalizations}
\begin{tabular}{lcp{3in}}
  \textbf{Option}&\textbf{Default}&\textbf{Description}\\
  \hline
  \texttt{n0\_norm} & $10^{14}$ & Density normalization (in cgs)\\
  \texttt{b0\_norm} & $10^4$    & Magnetic field normalization (in cgs)\\
  \texttt{l0\_norm} & $100$     & Length normalization (in cgs)
\end{tabular}


\subsection{Current Controller Options}

\begin{tabular}{llcl}
  \textbf{Option}&\textbf{Var.}&\textbf{Default}&\textbf{Description}\\
  \hline
  \texttt{vloop}      & $V_L$ & 0              & (Initial) loop voltage.\\
  \texttt{tcur}       & $I_0$ & \texttt{tcuro} & Target toroidal current\\
  \texttt{control\_p} & $c_p$ & 0              & Proportional coefficient\\
  \texttt{control\_i} & $c_i$ & 0              & Integral coefficient\\
  \texttt{control\_d} & $c_d$ & 0              & Derivative coefficient\\
\end{tabular}

\subsection{Density Source Options}

\begin{tabular}{llcp{2in}}
  \textbf{Option}&\textbf{Var.}&\textbf{Default}&\textbf{Description}\\
  \hline
  \texttt{ipellet}      & & 0    & 1: include pellet density
    source (\textit{c.f.} section~\ref{sec:pellet_injection})\\
  \texttt{pellet\_rate} & $\alpha_p$ & 0 
                                     & Particle number injection rate\\
  \texttt{pellet\_var}  & $l_p$      & 1    & Variance of  
                                              injection profile\\
  \texttt{pellet\_x}    & $R_p$      & \texttt{xmag} 
                                     & $R$-coordinate of injection profile\\
  \texttt{pellet\_z}    & $Z_p$      & \texttt{zmag} 
                                     & $Z$-coordinate of injection profile\\
  \texttt{ionization}   & & 0  & 1: include neutral ionization
    source (\textit{c.f.} section~\ref{sec:ionization})\\
  \texttt{ionization\_rate} & $\alpha_i$ & 0 
                                     & Ionization rate coefficient\\
  \texttt{ionization\_temp} & $E_i$   & 0.01 & Ionization energy\\
  \texttt{ionization\_depth}& $l_i$   & 0.01 & Temperature 
    scale-length of neutral burn-out
\end{tabular}

\subsection{Diagnostics Options}

\begin{tabular}{lcp{3in}}
  \textbf{Option}&\textbf{Default}&\textbf{Description}\\
  \hline
  \texttt{icalc\_scalars} & 1 & 1: Calculate volume-integrated scalars 
                                (\textit{e.g.} energy)\\
  \texttt{ike\_only} & 0 & 1: Only calculate kinetic energy
\end{tabular}
\subsection{Physical Parameters}

\begin{tabular}{llcl}
  \textbf{Option}&\textbf{Var.}&\textbf{Default}&\textbf{Description}\\
  \hline
  \texttt{gam}    & $\Gamma$& 5/3 & Adiabatic constant\\
  \texttt{db}     & $d_i$   & 0   & Ion skin depth\\ 
\end{tabular}


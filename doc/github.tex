\section{GITHUB}
Retrieve the current version of M3D-C1 from the GIT repository.
For the first time, to check out the sources, do:

Initial access is with the {\it clone} command.   This copies the source
code from the master file into a working directory on your machine.
You only do this once on each computer you work on.

{\it module load git}

{\it git clone https://github.com/PrincetonUniversity/M3DC1}

Subsequent GIT commands used to commit:
\begin{itemize}

\item {\it add/commit/push}:  You {\it add} files to a list of
files to update,
{\it commit} the chanes to your branch, and then {\it push}
the changes to the master branch.

\item {\it git commit -m ``message describing changes'' }
    (adding {\it -a} commits all changes)

\item {\it diff} lists the changes you made from the last commit,
    even if you haven'tpushed your commits to github.
    To see how your files differ from what's on github,
    you can do:

    \hspace{3 cm} {\it git fetch origin master}

    \hspace{3 cm} {\it git diff origin/master}

\item {\it status} compares your branch with the master branch

\item {\it pull origin master} updates your local branch to the
  current master branch

\item {\it stash} takes uncommitted changes, saves them for later use, and
  reverts files in working directory

  \hspace{2 cm} {\it stash list} \hspace{0.5 cm} {\it stash drop} \hspace{0.5 cm}
         {\it stash apply} \hspace{0.5 cm} {\it stash pop} (apply and drop)

\item {\it stash pop} removes changes from your stash and reapplies
  them to working copy

\item {\it stash apply} keeps changes in stash, but reapplies them to working copy

\item {\it reset -hard} discards any changes to local branch since last commit

\item {\it branch} tells you what branch you are in

\item {\it log} (--oneline) (-after 2023-01-31) lists all the commits
  for the checked-out branch after that date

\item {\it checkout hashtag} replaces your version with the version
  that has hastag ``hashtag''  (need only first 7 characters)
  
\end{itemize}

\subsection{Branches in GIT}

To make a new branch called fp-phase2:

\begin{itemize}

\item {\it git checkout master} switch to the master branch

\item {\it git pull}  make sure the master branch is up to date


\item {\it git checkout -b fp\_phase 2} The =b creates a new branch.
  This will be identical tomaster to start.

\item {\it git push --set-upstream origin fp\_phase2} push this new branch
to the remote so others can access it 

\end{itemize}


\subsection{Committing changes}

For example, to commit changes to newpar.f90

\begin{itemize}

\item {\it git pull} Always do thi before you start committing

\item {\it git add newpar.f90} This stages the current changes in newpar.f90
  for  commit.  You could then make more change before committing,
  but you would have to add again to get those into the commit.
  Or, you could add -a to the following to commit all changes.

\item {\it git commit -m ``Changed newpar.f90''}  Commit changes to
  your local branch.

\item {\it git push} Push commits to the remote repo.
  --set-upstream only needs to be done the first time.

\end{itemize}

\subsection{Merging branches}

To merge changes on master into fp\_phase2

\begin{itemize}

\item {\it git checkout master}  switch to the master branch

\item {\it git pull} get the latest commits on the master branch

\item {\it git checkout fp\_phase2}  switch back to fp\_phase2 
  

\item {\it git pull} to get the latest commits to fp\_phase2

\item {\it git merge master} merge any new commits into fp\_phase2.
  This makes a commit.  You may need to resolve conflicts.
  
\item {git push} Push the merged commit to remote repo

  
\end{itemize}

Inverting fb\_phase2 and master here would merge the development branch
into master locally, then the push wouuld send the merge tothe remote master

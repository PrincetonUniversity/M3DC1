\section{Obtaining M3D-C1}

%%%%%%%%%%%%%%%%%%%%%%%%%%%%%%%
\subsection{License Agreement}
%%%%%%%%%%%%%%%%%%%%%%%%%%%%%%%

M3D-C1 is a code developed with funding from the US Department of
Energy, and is intended for open scientific research.  If you intend
to run M3D-C1, please complete the following steps:

\begin{enumerate}
\item Sign the license agreement
  \url{https://m3dc1.pppl.gov/M3D-C1\_License.pdf} and return to
  \href{mailto:nferraro@pppl.gov}{nferraro@pppl.gov}.
\item Request access to the code using the form
  \url{https://pppl.tiny.us/code-release-form}.  This step will
  involve review by PPPL to clear any potential export control issues.
\end{enumerate}

Please be aware that permission to run M3D-C1 does not carry an
implicit agreement to provide technical support for compiling,
running, modifying, or interpreting output of, M3D-C1.


%%%%%%%%%%%%%%%%%%%%%%%%%%%%%%%
\subsection{Preinstalled M3D-C1 Executables}
%%%%%%%%%%%%%%%%%%%%%%%%%%%%%%%

For the general user, we recommend using precompiled executables and
associated modules for release versions, where available.
Installations are presently available on a number of systems,
including:

\begin{description}
\item{GA Iris}:\\
  \texttt{module use /fusion/projects/codes/m3dc1/modules}\\
  \texttt{module load m3dc1/1.13}
\item{NERSC Cori}:\\
  \texttt{module use /global/cfs/projectdirs/mp288/C1/modules/cori}\\
  \texttt{module load m3dc1/1.13-haswell}\quad
  for Phase I (Haswell) partition\\
  \texttt{module load m3dc1/1.13-knl}\quad
  for Phase II (KNL) partition
\item{PPPL Cluster}:\\
  \texttt{module use /p/m3dc1/modules}\\
  \texttt{module load m3dc1/1.13}
\item{PU Stellar}:\\
  \texttt{module use /projects/M3DC1/modules}\\
  \texttt{module load m3dc1/1.13}
\end{description}

These modules will modify the user's enviroment variables
appropriately to access the M3D-C1 executables, python libraries, and
IDL routines.



%%%%%%%%%%%%%%%%%%%%%%%%%%%%%%%
\subsection{Accessing the M3D-C1 Source Code}
%%%%%%%%%%%%%%%%%%%%%%%%%%%%%%%

If you choose to build the code yourself, either to use an unreleased
version or to do code development yourself, you will need access to
the M3D-C1 code repository.  The M3D-C1 source code is located in the
Github repository: \texttt{PrincetonUniversity/M3DC1}. Access to this
repository requires a Github account and permission from
\href{mailto:nferraro@pppl.gov}{Dr. Nate Ferraro}.


%%%%%%%%%%%%%%%%%%%%%%%%%%%%%%%
\subsection{Makefiles and Dependencies}
%%%%%%%%%%%%%%%%%%%%%%%%%%%%%%%

Some of the build scripts depend on the following environment
variables to be set, to specify the location of the M3D-C1 source code
and the directory in which to install any compiled executables:
\begin{description}
\item{\texttt{M3DC1\_DIR}} should be set to the directory containing
  the M3D-C1 source code.  For example:\\
  \texttt{setenv M3DC1\_DIR \$HOME/src/M3DC1}
\item{\texttt{M3DC1\_INSTALL\_DIR}} should be set to the directory in
  which M3D-C1 will be installed.  For example:\\
  \texttt{setenv M3DC1\_INSTALL\_DIR \$HOME/M3DC1}
\end{description}
It is recommended to set these values in your login script.

Makefiles for a number of systems are included in the repository,
with filenames \texttt{\$M3DC1\_DIR/unstructured/*.mk}.  For most of
these systems, environment modules are also included in the
repository.  These modules will load the appropriate software modules
for building on that particular system, and can be loaded using:

\begin{description}
\item{NERSC Cori}:\\
  \texttt{module use \$M3DC1\_DIR/modules/cori}\\
  \texttt{module load m3dc1/devel-haswell}\quad
  for Phase I (Haswell) partition\\
  \texttt{module load m3dc1/devel-knl}\quad
  for Phase II (KNL) partition
\item{NERSC Perlmutter}:\\
  \texttt{module use \$M3DC1\_DIR/modules/perlmutter}\\
  \texttt{module load m3dc1/devel}
\item{PPPL Cluster}:\\
  \texttt{module use \$M3DC1\_DIR/modules/pppl}\\
  \texttt{module load m3dc1/devel-centos7}
\item{PU Stellar}:\\
  \texttt{module use \$M3DC1\_DIR/modules/stellar}\\
  \texttt{module load m3dc1/devel}
\end{description}

It is recommended to place the appropriate \texttt{module use}
statement in your login script.  The makefile should have the name
\texttt{\$\{M3DC1\_ARCH\}.mk}.  If \$M3DC1\_ARCH is not defined, it
will default to \texttt{\$HOST}, stripped of any trailing numbers
identifying a node index on multinode systems (\textit{e.g.} if
\texttt{\$HOST==``sunfire06''} then \texttt{\$M3DC1\_ARCH} will
default to ``sunfire'').

If you are building M3D-C1 on a system for which no makefile or development
module is provided, please refer to the existing makefiles and modules
as examples.  In general, M3D-C1 requires the following dependencies:
\begin{itemize}
\item Compilers for C, C++, and Fortran;
\item MPI
\item HDF5 compiled with support for Fortran and MPI
\item netcdf
\item GSL
\item FFTW
\item PETSc compiled with support for Fortran, complex-valued
  functions, MUMPS and/or SuperLU\_dist
\item PUMI meshing libraries
\end{itemize}

%%%%%%%%%%%%%%%%%%%%%%%%%%%%%%%
\subsection{Building}
%%%%%%%%%%%%%%%%%%%%%%%%%%%%%%%

Once the appropriate makefile has been defined, the M3D-C1 executables
can be built by entering \texttt{\$M3DC1\_DIR/unstructured} and
running

\texttt{make all}

This will run the following commands:

\begin{tabular}{ll}
\texttt{make OPT=1} & Builds the 2D version\\
\texttt{make OPT=1 COM=1} & Builds the complex version\\
\texttt{make OPT=1 3D=1 MAX\_PTS=60} & Builds the 3D version\\
\texttt{make OPT=1 3D=1 MAX\_PTS=60 ST=1} & Builds the stellarator version\\
\texttt{make a2cc} & Builds a utility for extracting coil currents
from a-eqdsk files\\
\texttt{make bin} & Puts executables into the \texttt{\_\$M3DC1\_ARCH} subdirectory\\
\end{tabular}


%%%%%%%%%%%%%%%%%%%%%%%%%%%%%%%
\subsection{Regression Tests}
%%%%%%%%%%%%%%%%%%%%%%%%%%%%%%%

The M3D-C1 source code includes a suite of regression tests that
should be run before commiting any new code to the repository.  To run
these tests:

\texttt{cd \$M3DC1\_DIR/unstructured/regtest}
\texttt{./run <arch> <test>}

where \texttt{<arch>} and \texttt{<test>} are optional arguments
specifying the specific batch script to run, and the specific
regression test to run, respectively.  By default,
\texttt{<arch>=\$M3DC1\_ARCH}.

If \texttt{<test>} is not
specified, then all the regression tests will be run, using the batch
scripts\\
\texttt{\$M3DC1\_DIR/unstructured/regtest/*/base/batchjob.<arch>}.

The \texttt{run} script will create new directories in which to run
these regression tests, named\\
\texttt{\$M3DC1\_DIR/unstructured/regtest/*/\$\{M3DC1\_VERSION\}\_<arch>/}.

If \texttt{<test>} is specified, then only
\texttt{\$M3DC1\_ARCH/unstructured/regtest/<test>/base/batchjob.<arch>}
will be run (again, in a new directory named as described above).  If
\texttt{<test>} is specified, then \texttt{<arch>} must also be
specified.

To check the results of the regression tests

\texttt{./check <arch>}


